\documentclass[12pt]{article}%{revtex4}
\usepackage[a4paper, left=1.5cm, right=1.5cm, top=2cm, bottom=2cm]{geometry}


\usepackage{bbold}
\usepackage{amsmath}
\usepackage{amssymb}
\usepackage{graphicx}
\usepackage{float}
\usepackage{subfigure}
\usepackage{caption}
\usepackage{physics} 
\usepackage{tcolorbox}
\usepackage{import}
\usepackage{tikz}
\usepackage{pgfplots}
\usepackage{subfiles}

\usepackage{fontspec}
\usepackage{xeCJK}
\setCJKfamilyfont{kai}{標楷體}

%================================================================================================
\begin{document}
\begin{center}
\textbf{\Large Introducting Relativity}
\\\vspace{10pt}
{Problem Set 1 (Due 2024/3/12)}
\\\vspace{10pt}
Chang-Mao Yang, 楊長茂(409220055), \today

\vspace{5pt}
\hrule
\end{center}
%-----------------------------------------------------------------------------------


\begin{enumerate}

%-----------------
\item (Compton scattering 20\%) A photon of wave length $\lambda$ is scattered by a rest electron of mass $m_e$. If the scattered photon has wave length $\lambda'$ with angle $\theta$ from the incident direction. Show that
\begin{equation*}
\lambda' - \lambda = \frac{h}{m_ec} \left(1-\cos\theta\right).
\end{equation*}
(Hint1. The 4-momentum $p^{\mu} = (E/c, \vec{p}\,)$ satisfies $p^{\mu}p_{\mu} = (E/c)^2 - \vec{p}\,^2 = m_0c^2$.)\\
(Hint2. The Energy and momentum for photon are given by $E = pc=h\nu$ and $c=\lambda \nu$)

\noindent\fbox{\begin{minipage}[c]{0.98\linewidth}
%---
\begin{center}
\subfile{img/Fig1.tex}
\end{center}

Since, the momentum of the photon is $p = h\nu/c = h/\lambda$
The 4-momentum of the photon $\gamma$ and electron $e$ are 
\begin{center}
\begin{tabular}{l|c|c}
	& Before & After\\\hline\hline
photon $\gamma$	
& $\displaystyle p_{\gamma}^{\mu} = \left(\frac{h}{\lambda}, \frac{h}{\lambda}, 0, 0\right)$ 
& $\displaystyle  p'^{\mu}_{\gamma} = \left(\frac{h}{\lambda'}, \frac{h}{\lambda'}\cos\theta, \frac{h}{\lambda'}\sin\theta, 0\right)$\\\hline
electron $e$	& $p^{\mu}_{m_e} = (m_e c, 0, 0, 0)$ & $p'^{\mu}_{m_e}=\text{unknown}$
\end{tabular}
\end{center}
Then using the conservation of 4-momentum, we have$p^{\mu}_{\gamma} + p^{\mu}_{m_e} = p'^{\mu}_{\gamma} + p'^{\mu}_{m_e}$, then rearranging the equation $p'^{\mu}_{m_e} = p^{\mu}_{\gamma} + p^{\mu}_{m_e} - p'^{\mu}_{\gamma}$.
Using the properties of momentum for the $p'^{\mu}_{m_e}$, i.e. $(p'_{m_e})^{\mu}(p'_{m_e})_{\mu} = m_e^2 c^2$, so that
\begin{align}
(p'_{m_e})^{\mu}(p'_{m_e})_{\mu} 
&= \left[(p_{\gamma})^{\mu} + (p_{m_e})^{\mu} - (p'_{\gamma})^{\mu}\right]
	\left[(p_{\gamma})_{\mu} + (p_{m_e})_{\mu} - (p'_{\gamma})_{\mu}\right],\\
&= (p_{\gamma})^{\mu} (p_{\gamma})_{\mu} 
	+ (p_{m_e})^{\mu} (p_{m_e})_{\mu}
	+ (p'_{\gamma})^{\mu} (p'_{\gamma})_{\mu}\\
&\quad +(p_{\gamma})^{\mu}(p_{m_e})_{\mu} - (p_{\gamma})^{\mu}(p'_{\gamma})_{\mu}
 +(p_{m_e})^{\mu}(p_{\gamma})_{\mu} - (p_{m_e})^{\mu}(p'_{\gamma})_{\mu}\\
&\quad -(p'_{\gamma})^{\mu}(p_{\gamma})_{\mu} - (p'_{\gamma})^{\mu}(p_{m_e})_{\mu},\\
&= (p_{\gamma})^{\mu} (p_{\gamma})_{\mu} 
	+ (p_{m_e})^{\mu} (p_{m_e})_{\mu}
	+ (p'_{\gamma})^{\mu} (p'_{\gamma})_{\mu}\\
&\quad +2(p_{\gamma})^{\mu}(p_{m_e})_{\mu} 
	- 2(p_{\gamma})^{\mu}(p'_{\gamma})_{\mu}
 	- 2(p_{m_e})^{\mu}(p'_{\gamma})_{\mu}.
\end{align}
Plugin all the values of 4-momentum, we have
\begin{equation}
m_e^2 c^2 
= 0 + m_e^2 c^2 + 0 + 2\frac{h}{\lambda} m_ec - 2\frac{h^2}{\lambda\lambda'}\left(1-\cos\theta\right) - 2m_ec\frac{h}{\lambda'}.
\end{equation}
Rearranging the equation
\begin{align}
0 &= \frac{h}{\lambda} m_ec - \frac{h^2}{\lambda\lambda'}\left(1-\cos\theta\right) - m_ec\frac{h}{\lambda'},\\
\frac{\lambda\lambda'}{hm_ec}\cdot 0 = 0 &= \lambda' - \frac{h}{m_ec}\left(1-\cos\theta\right) - \lambda,
\end{align}
That is $\displaystyle \lambda - \lambda' = \frac{h}{m_ec}\left(1-\cos\theta\right)$.






%---
\end{minipage}}

\newpage
%-----------------
\item (Relativistic energy 20\%) Using the relativistic force defined by
\begin{equation*}
\vec{F} = \frac{d}{dt}\left(\gamma m_0 \vec{u}\right),\quad \gamma = \frac{1}{\sqrt{1-u^2/c^2}}
\end{equation*}
and the work done by $\vec{F}$ as 
\begin{equation*}
\frac{dE}{dt} = \vec{F}\cdot \vec{u}
\end{equation*}
to show that
\begin{equation*}
E = \frac{m_0c^2}{\sqrt{1-u^2/c^2}} + \mathrm{constant}.
\end{equation*}

\noindent\fbox{\begin{minipage}[c]{0.98\linewidth}
%---
Using the work done by $\vec{F}$, we can solve the energy by integrating the equation 
\begin{align}
E &= \int \frac{dE}{dt}dt = \int \vec{F}\cdot \vec{u} dt\\
&= \int \frac{d}{dt}\left(\gamma m_0 \vec{u}\right) \cdot \vec{u} dt
= \int \vec{u} \cdot \frac{d\left(\gamma m_0 \vec{u}\right)}{dt} dt\\
& = \int \vec{u} \cdot \frac{d\left(\gamma m_0 \vec{u}\right)}{dt} dt
= \int \vec{u} \cdot d\left(\gamma m_0 \vec{u}\right)\\
& = \int \vec{u} \cdot \left(\gamma m_0 d \vec{u} +  m_0 \vec{u} d\gamma\right)\\
&= \int \gamma m_0\vec{u}\cdot d\vec{u} +  \int m_0\vec{u}\cdot\vec{u}\,d\gamma \\
&= \int \gamma m_0\vec{u}\cdot d\vec{u}  +  \int m_0u^2d\left(1-u^2/c^2\right)^{-1/2}\\
&= \int \gamma m_0\vec{u}\cdot d\vec{u}  + \int m_0 u^2 \frac{-1}{2}\left(1-u^2/c^2\right)^{-3/2} d\left(-u^2/c^2\right)\\
&= \int \gamma m_0\vec{u}\cdot d\vec{u}  + \int \frac{1}{2} m_0 u^2 \gamma^3 d\left(u^2/c^2\right)\\
&= \int \gamma m_0\vec{u}\cdot d\vec{u}  + \int \frac{1}{2} m_0 u^2 \gamma^3 \frac{2\vec{u}}{c^2}\cdot d\vec{u}\\
&= m_0 \int  \left( \gamma  +  \gamma^3 \frac{u^2}{c^2}\right)\vec{u}\cdot d\vec{u}
= m_0 \int  \gamma^3 \left( \gamma^{-2}  + \frac{u^2}{c^2}\right)\vec{u}\cdot d\vec{u}\\
&= m_0 \int  \gamma^3 \left( \left(1-\frac{u^2}{c^2}\right)  + \frac{u^2}{c^2}\right)
\vec{u}\cdot d\vec{u}
= m_0 \int  \gamma^3 \vec{u}\cdot d\vec{u}\\
&= m_0 \int  \left(1-\frac{u^2}{c^2}\right)^{-3/2} du^2/2
= \frac{-m_0 c^2}{2}\int  \left(1-\frac{u^2}{c^2}\right)^{-3/2} d(-u^2/c^2)\\
&= -\frac{m_0c^2}{2} \cdot \left(1-\frac{u^2}{c^2}\right)^{-1/2} \cdot (-2) + \mathrm{constant.}\\
&= \frac{m_0c^2}{\sqrt{1-u^2/c^2}} + \mathrm{constant}.
\end{align}
%---
\end{minipage}}




%-----------------
\newpage
\item (Vector fields 60\%) In $\mathbb{R}^2$ a point can be expressed in Cartesian coordinate $(x^{a}) = (x,y)$ or polar coordinate $(x'^{a}) = (r,\theta)$.
\begin{enumerate}
\item[(1a)] Find the transformation matrix $J' = (\partial x'^{a}/\partial x^{b})$ and its inverse $J = (\partial x^{a}/\partial x'^{b})$ in terms of $x'^{a}$.
\item[(1b)] Let $f(x,y)$ be a function on the circle $x^2 + y^2 = a^2$. Obtain the vector field $X = X^{a}\partial_{a}$ such that $df/d\theta = Xf$.
\item[(1c)] Using $J'$ to obtain the corresponding $X'$ a for the vector field in (1b).
\end{enumerate}



\noindent\fbox{\begin{minipage}[c]{0.98\linewidth}
%---
\begin{enumerate}
\item[(1a)] Notice that
\begin{equation}
x^{a}(x'^{a}) = \begin{pmatrix}x(r,\theta)\\y(r,\theta)\end{pmatrix} 
= \begin{pmatrix}
r\cos\theta\\
r\sin\theta
\end{pmatrix},
\end{equation}
so 
\begin{equation}
dx^{a} = \begin{pmatrix}dx\\dy\end{pmatrix} 
= \begin{pmatrix}
dr\cos\theta - r\sin\theta d\theta\\
dr\sin\theta + r\cos\theta d\theta
\end{pmatrix}
= \begin{pmatrix}
\cos\theta & -r\sin\theta\\
\sin\theta & r\cos\theta
\end{pmatrix}\begin{pmatrix}dr\\d\theta\end{pmatrix} 
= \frac{dx^{a}}{dx'^{b}} dx'^{b},
\end{equation}
which means the Jacobian $J = (\partial x^{a}/\partial x'^{b})$ is 
\begin{equation}
J = \frac{dx^{a}}{dx'^{b}} = \begin{pmatrix}
\cos\theta & -r\sin\theta\\
\sin\theta & r\cos\theta
\end{pmatrix}.
\end{equation}
And its determinant is $\det(J) = r$, then we have the inverse transformation 
\begin{equation}
J' = \frac{dx'^{a}}{dx^{b}} = \frac{1}{\det(J)}\begin{pmatrix}
r\cos\theta & r\sin\theta\\
-\sin\theta & \cos\theta
\end{pmatrix} 
= \begin{pmatrix}
\cos\theta & \sin\theta\\
-\sin\theta/r & \cos\theta/r
\end{pmatrix} .
\end{equation}

\item[(1b)] For the function $f(x,y)$ on the circle $x^2 + y^2 = a^2$, we can using a new coordinate $x'^{a} = (r,\theta)$ as same as (1a), i.e. $x=r\cos\theta$ and $y=r\sin\theta$, where $r=a$. So that 
\begin{equation}
\frac{df}{d\theta}
= \frac{\partial x}{\partial \theta} \frac{\partial f}{\partial x} + \frac{\partial y}{\partial \theta} \frac{\partial f}{\partial y}
= \left(\frac{\partial x}{\partial \theta} \frac{\partial}{\partial x} + \frac{\partial y}{\partial \theta} \frac{\partial}{\partial y}\right)f
= \left(-a\sin\theta\frac{\partial}{\partial x} + a\cos\theta\frac{\partial}{\partial y}\right)f
\end{equation}
denote as
\begin{equation}
\frac{df}{d\theta} = Xf = X^{a}\partial_{a} f,
\end{equation}
where $X$ is the vector field given by
\begin{equation}
X = -a\sin\theta\frac{\partial}{\partial x} + a\cos\theta\frac{\partial}{\partial y}
\end{equation}



\item[(1c)] For a vector field in (1b), we can using the transformation to obtain
\begin{align}
X'^{a} &= \frac{\partial x'^{a}}{\partial x^{b}}X^{b}\\
&= \begin{pmatrix}
\cos\theta & \sin\theta\\
-\sin\theta/a & \cos\theta/a
\end{pmatrix}
\begin{pmatrix}
-a\sin\theta\\a\cos\theta
\end{pmatrix}\\
&= \begin{pmatrix}
-a\sin\theta\cos\theta + a\cos\theta\sin\theta\\
\sin^2\theta +  \cos^2\theta
\end{pmatrix} = \begin{pmatrix}
0 \\ 1
\end{pmatrix}.
\end{align}



\end{enumerate}
%---
\end{minipage}}












\end{enumerate}

%-----------------------------------------------------------------------------------
\end{document}