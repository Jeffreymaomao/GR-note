\documentclass[12pt]{article}%{revtex4}
\usepackage[a4paper, left=1.5cm, right=1.5cm, top=2cm, bottom=2cm]{geometry}


\usepackage{bbold}
\usepackage{amsmath}
\usepackage{amssymb}
\usepackage{graphicx}
\usepackage{float}
\usepackage{subfigure}
\usepackage{caption}
\usepackage{physics} 
\usepackage{tcolorbox}
\usepackage{import}
\usepackage{tikz}
\usepackage{pgfplots}
\usepackage{subfiles}

\usepackage{fontspec}
\usepackage{xeCJK}
\setCJKfamilyfont{kai}{標楷體}

%================================================================================================
\begin{document}
\begin{center}
\textbf{\Large Introducting Relativity}
\\\vspace{10pt}
{Problem Set 1 (Due 2024/3/12)}
\\\vspace{10pt}
Chang-Mao Yang, 楊長茂(409220055), \today

\vspace{5pt}
\hrule
\end{center}
%-----------------------------------------------------------------------------------

\begin{figure}[h]
\centering
\subfile{img/Fig1.tex}
\caption{Light cone and causality of the frame $S$}
\end{figure}

\begin{enumerate}

%-----------------
\item (Causality)
Let $E$ be the event on the light-cone and $G$ be an event outside light-cone in the inertia frame $S$.
	\begin{enumerate}
	\item Draw the axis of the frame $S'$ in $S$ so that $G$ occurs earlier than $O$.
	\item Draw the axis of the frame $S''$ in $S$ so that $G$ occurs later than $O$.	\end{enumerate}

\noindent\fbox{\begin{minipage}[c]{0.98\linewidth}
%---
The 2 drawing figure see as Figure \ref{fig:axis in S}.
\begin{center}
\subfile{img/Fig2.tex}
\captionof{figure}{Axes of the frame $S'$ and $S''$ in $S$}
\label{fig:axis in S}
\end{center}
%---
\end{minipage}}

\newpage
%-----------------
\item (Lorentz transformation) Given the Lorentz transformation in $x$-direction
\begin{equation*}
\begin{pmatrix}ct'\\x'\end{pmatrix}
= 
\begin{pmatrix}\gamma&-\beta\gamma\\-\beta\gamma&\gamma\end{pmatrix}
\begin{pmatrix}ct\\x\end{pmatrix}
\equiv L(v)\begin{pmatrix}ct\\x\end{pmatrix}.
\end{equation*}
Show that from the combination of two Lorentz transformations $L(v_2)L(v_1) = L(v)$ one has
\begin{equation*}
v = \frac{v_1+v_2}{1+v_1v_2/c^2}.
\end{equation*}

\noindent\fbox{\begin{minipage}[c]{0.98\linewidth}
%---
For two different velocities $v_1$ and $v_2$, we define $\displaystyle\gamma_i = \frac{1}{\sqrt{1-\beta_i^2}}$ where $\displaystyle \beta_i = \frac{v_i}{c}$, $i=1,2$. Also, same for the addition velocity $v$ we define $\displaystyle \gamma = \frac{1}{\sqrt{1-\beta^2}}$, where $\displaystyle\beta = \frac{v}{c}$.

For convenience, we first compute
\begin{align}
\gamma_1\gamma_2 
&= \frac{1}{\sqrt{1-\beta_1^2}}\frac{1}{\sqrt{1-\beta_2^2}},\\
&= \frac{1}{\sqrt{1+\beta_1^2\beta_2^2-(\beta_1^2+\beta_2^2)}},\\
&= \frac{1}{\sqrt{(1+2\beta_1\beta_2+\beta_1^2\beta_2^2)-(\beta_1^2+2\beta_1\beta_2+\beta_2^2)}},\\
&= \frac{1}{\sqrt{(1+\beta_1\beta_2)^2 -(\beta_1+\beta_2)^2}},\\
&= \frac{1}{1+\beta_1\beta_2}\frac{1}{\sqrt{1 -(\beta_1+\beta_2)^2/(1+\beta_1\beta_2)^2}}.
\end{align}
Now, we the product of two matrix $L(v_1)$ and $L(v_2)$ is
\begin{align}
L(v_1)L(v_2) 
&= \begin{pmatrix}
	\gamma_1			&	-\beta_1\gamma_1\\
	-\beta_1\gamma_1	&	\gamma_1
	\end{pmatrix}
	\begin{pmatrix}
	\gamma_2			&	-\beta_2\gamma_2\\
	-\beta_2\gamma_2	&	\gamma_2
	\end{pmatrix}\\
&= \begin{pmatrix}
	\gamma_1\gamma_2 + \beta_1\gamma_1\beta_2\gamma_2 
	&
	-\beta_2\gamma_2\gamma_1 -\beta_1\gamma_1\gamma_2
	\\
	-\beta_2\gamma_2\gamma_1 -\beta_1\gamma_1\gamma_2
	&
	\gamma_1\gamma_2 + \beta_1\gamma_1\beta_2\gamma_2
	\end{pmatrix}\\
&= \gamma_1\gamma_2 \begin{pmatrix}
	1 + \beta_1\beta_1
	&
	-(\beta_1 + \beta_2)
	\\
	-(\beta_1 + \beta_2)
	&
	\beta_1\beta_2 + 1
	\end{pmatrix}\\
&= \frac{1}{1+\beta_1\beta_2}\frac{1}{\sqrt{1 -(\beta_1+\beta_2)^2/(1+\beta_1\beta_2)^2}}
\begin{pmatrix}
	1 + \beta_1\beta_1
	&
	-(\beta_1 + \beta_2)
	\\
	-(\beta_1 + \beta_2)
	&
	\beta_1\beta_2 + 1
	\end{pmatrix}\\
&= \frac{1}{\sqrt{1 -(\beta_1+\beta_2)^2/(1+\beta_1\beta_2)^2}}
\begin{pmatrix}
	1&\displaystyle -\frac{\beta_1 + \beta_2}{1+\beta_1\beta_2}
	\\
	\displaystyle-\frac{\beta_1 + \beta_2}{1+\beta_1\beta_2}&1.
	\end{pmatrix}
\end{align}
Ones we define $L(v) = \begin{pmatrix}\gamma&-\beta\gamma\\-\beta\gamma&\gamma\end{pmatrix} = \gamma\begin{pmatrix}1&-\beta\\-\beta&1\end{pmatrix} = L(v_1)L(v_2)$, we compare the result, yield that
\begin{equation}
\gamma = \frac{1}{\sqrt{1-\beta^2}},\quad\text{ where } \beta = \frac{\beta_1+\beta_2}{1+\beta_1\beta_2}.
\end{equation}
This result shows that the addition velocity is $\displaystyle v=c\beta = \frac{c\beta_1+c\beta_2}{1+\beta_1\beta_2} = \frac{v_1+v_2}{1+v_1v_2/c^2}$.
%---
\end{minipage}}




%-----------------
\newpage
\item (Length contraction) Let $S$ and $S'$ are inertial frames relative with $v = \alpha c$ where $0<\alpha<1$. If a rod at rest in $S'$ makes an angle of $\pi/6$ with $Ox'$ in $S'$ and $\pi/4$ with $Ox$ in $S$. Find the parameter $\alpha$.

\noindent\fbox{\begin{minipage}[c]{0.98\linewidth}
%---
Define the length of rod to be $L$ in $S$ and $L'$ in $S'$. The length projection on $x,y$-axis in $S$ and $x,y$-axis in $S'$ are given by
\begin{equation}
\begin{cases}
L_x = L \cos(\pi/4) = L\sqrt{2}/2\\
L_y = L \sin(\pi/4) = L\sqrt{2}/2
\end{cases}\quad\text{and}\quad
\begin{cases}
L'_{x} = L'\cos(\pi/6) = L'\sqrt{3}/2\\
L'_{y} = L'\sin(\pi/6) = L'/2
\end{cases}.
\end{equation}

\begin{center}
\subfile{img/Fig3.tex}
\captionof{figure}{Length contraction}
\label{fig:length}
\end{center}

Also, notice that the projection on $y$ are the same $L_y = L_y'$(see Figure \ref{fig:length}), so we may get 
\begin{equation}
L_y = L'_y \quad\Rightarrow\quad L\sqrt{2}/2 = L'/2 \quad\Rightarrow\quad \sqrt{2}L = L'.
\label{eq:length relation}
\end{equation}

According to the length contraction, we have the relation of length projection between two frames $S$ and $S'$, that is 
\begin{equation}
L_x' = \gamma(v)\,L_x = \frac{1}{\sqrt{1-v^2/c^2}}\,L_x = \frac{1}{\sqrt{1-\alpha^2}}\,L_x.
\end{equation}
Plugin the length in two frame and using the the relation of two length (\ref{eq:length relation}), we have
\begin{equation}
\frac{1}{\sqrt{1-\alpha^2}}
= \frac{L'_x}{L_x} 
= \frac{L'\sqrt{3}/2}{L\sqrt{2}/2} 
= \frac{L\sqrt{2}\sqrt{3}/2}{L\sqrt{2}/2} 
= \sqrt{3}
\end{equation}
solving the value of $\alpha$, 
\begin{equation}
\sqrt{1-\alpha^2} = \frac{1}{\sqrt{3}}
\quad\Rightarrow\quad
1-\alpha^2 = \pm \frac{1}{3}
\quad\Rightarrow\quad
\alpha^2 = 1 \pm \frac{1}{3}
\quad\Rightarrow\quad
\alpha = \pm \sqrt{1\pm \frac{1}{3}},
\end{equation}
we have the all possible values for $\alpha$, that is $\alpha=\pm \sqrt{1/2}, \pm \sqrt{3/2}$. However, we must have $0<\alpha<1$, so $\alpha=\sqrt{2/3}$.
%---
\end{minipage}}




%-----------------
\newpage
\item (Aberration)A light ray from a star to a telescope observer has an inclination $\theta'$ to the horizontal in $S'$ and $\theta$ in $S$, where $S$ and $S'$ are related by speed $v$. Show that 
\begin{equation*}
\tan\theta' = \frac{\sin\theta}{\gamma(\cos\theta+v/c)},
\end{equation*}
where $\gamma = 1/\sqrt{1-v^2/c^2}$.

\end{enumerate}

\noindent\fbox{\begin{minipage}[c]{0.98\linewidth}
%---
\begin{center}
\subfile{img/Fig4.tex}
\captionof{figure}{Rest frame and telescope frame}
\label{fig:length}
\end{center}

Using the addition formulae of velocity, we have 
\begin{equation}
u'_x=\frac{u_x+v}{1+u_xv/c^2}
\quad\text{and}\quad
u'_y=\frac{u_y}{\gamma(1+u_xv/c^2)}.
\end{equation}
So the tangent value angle $\theta'$ of the beam in $S'$ is 
\begin{equation}
\tan\theta' = \frac{u'_y}{u'_x}
= \frac{\displaystyle\frac{u_y}{\gamma(1+u_xv/c^2)}}{\displaystyle\frac{u_x+v}{1+u_xv/c^2}}
= \frac{u_y}{\gamma (u_x+v)}
= \frac{c\sin\theta}{\gamma (c\cos\theta+v)}
= \frac{\sin\theta}{\gamma (\cos\theta+v/c)}.
\end{equation}


%---
\end{minipage}}


































%-----------------------------------------------------------------------------------
\end{document}