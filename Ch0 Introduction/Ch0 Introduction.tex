
% Sets the document class and font size
\documentclass[12pt]{article}
\usepackage[a4paper, margin=1in]{geometry}

\usepackage[utf8]{inputenc}    % Input encoding
\usepackage[T1]{fontenc}       % Font encoding
\usepackage{fontspec}          % font encoding

% Advanced math typesetting
\usepackage{amsmath}
\usepackage{amssymb}
\usepackage{mathtools}
\usepackage{physics}

% Symbols and Text
\usepackage{bbold}     % bold font
\usepackage{ulem}      % strikethrough
\usepackage{listings}  % Source code listing
\usepackage{import}    % Importing code and other documents

% graphics
\usepackage[dvipsnames]{xcolor}
\usepackage{graphicx}
\usepackage{tikz}
\usepackage{pgfplots}
\usepackage{tcolorbox}

% figure
\usepackage{float}
\usepackage{subfigure}

\usepackage{hyperref}  % Hyperlinks in the document
\hypersetup{
    colorlinks=true,
    linkcolor=Blue,
    filecolor=red,
    urlcolor=Blue,
    citecolor=blue,
    pdftitle={Article},
    pdfauthor={Author},
}

\usepackage{xeCJK}             % Chinese, Japanese, and Korean characters
\setCJKfamilyfont{kai}{標楷體}
%================================================================================================
\begin{document}

\subsection{Ch.0 Introduction}

<u>Fig 1 - theory cube</u>

\textbf{Dimension Analysis}

\begin{itemize}
	\item The physical constants

\end{itemize}
	\begin{itemize}
		\item Speed of light: $\displaystyle [c] = \frac{L}{T}$

	\end{itemize}
	\begin{itemize}
		\item Gravitational constant: $\displaystyle [G] = \frac{L^3}{MT^2}$

	\end{itemize}
	\begin{itemize}
		\item Planck's constant: $\displaystyle [\hbar] = \frac{ML^2}{T}$

	\end{itemize}
\begin{itemize}
	\item The quantities
	\begin{itemize}
		\item Planck's Length: $\displaystyle \ell_P = \sqrt{\frac{G\hbar}{c^3}} \simeq 10^{-33}\,\mathrm{cm}$​
		\item Planck's Time: $\displaystyle t_P = \sqrt{\frac{G\hbar}{c^5}} \simeq 10^{-44}\,\mathrm{sec}$
		\item Planck's Mass: $\displaystyle M_P = \sqrt{\frac{c\hbar}{G}}\simeq 10^{19}\,\mathrm{GeV/c^2}$

	\end{itemize}
\textbf{Fundamental forces}

\end{itemize}
\begin{itemize}
	\item \textit{EM force} (matter \& photon)
	\item \textit{weak force} (nuclei decay): slow process
	\item \textit{strong force} (binding quarks)
	\item \textit{gravitational force} (massive particles)

\end{itemize}
<u>Fig 2 - theory tree</u>

unification of fundamental force

<u>Fig 3 - forces unification</u>

\textbf{Equations}

\begin{itemize}
	\item Newton's laws : $\vec{F} = m\vec{a}$  (1 eqs)

\end{itemize}
\begin{itemize}
	\item Masswell eqution (2x1+2x3 eqs)

\end{itemize}
\begin{itemize}
	\item Schodinger equation (1 eqs)

\end{itemize}
\begin{itemize}
	\item Einstein's equation $G_{\mu\nu} = \kappa T_{\mu\nu}$​  (4x4 eqs)

\end{itemize}
	\begin{itemize}
		\item $G_{\mu\nu}$ : Einstein tensor (Geometry of spacetime)
		\item $T_{\mu\nu}$​ : Energy-Momentum tensor

	\end{itemize}
\begin{quote}
	John Wheeler (1911~2008)\\
$\longrightarrow$ \textit{Curved spacetime tells the matter how to move}\\
$\longleftarrow$ \textit{Matter tells the spacetime how to curved}
\end{quote}

\textbf{Notions}

In our course, the Minkowski metric
\begin{equation}
\eta_{\mu\nu} = \operatorname{diag}([1,-1,-1,-1])
\end{equation}
in other (like Special Relativity) $\eta_{\mu\nu} = \operatorname{diag}([-1,1,1,1])$




%================================================================================================
\end{document}
