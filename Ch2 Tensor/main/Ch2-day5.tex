\begin{quote}
	\textbf{Review}: Lie dedrivative
\begin{equation}
\lim_{\delta u \to 0} \frac{T(Q)-T(P)}{\delta u}
\end{equation}
\underline{Fig}
Drag $T(P)\to T'(Q)$ then define $\displaystyle L_X T = \lim_{\delta u\to 0} \frac{T(Q)-T'(Q)}{\delta u}$, where the drag
\begin{equation}
T'^{a}(x+\delta u\, X) = \left(\frac{\partial x'^{a}}{\partial x^{b}}\right) T^{b}(x)
\end{equation}
So that the Lie dedrivative  $L_X = X^{b}\partial_{b}T^{a} - T^{b}\partial_b X^{a}$, which is a contravariant vector.
In general, for $T^{a_1,a_2,\cdots,a_n}$, the Lie dedrivative is defined to be
\begin{equation}
L_X T^{a_1,a_2,\ldots,a_n} = X^{b}\partial_b T^{a_1,a_2,\ldots,a_n} - \left(\right)
\end{equation}

\end{quote}
\begin{quote}
	Rmk:
\begin{equation}
\begin{aligned}
L_X Y^{a} &= X^{b}\partial_b Y^{a} - Y^{b}\partial_b X^{a}\\
&= [X,Y]^{a}
\end{aligned}
\end{equation}

\end{quote}
\begin{quote}
	Rmk:
For scalar $\phi$, its drag is $\phi'(x') = \phi(x)$, so
\begin{equation}
L_{X} \phi = \lim_{\delta u\to 0}\frac{\phi(X+\delta u) - \phi(X)}{\delta u} = X^{c}\partial_c \phi
\end{equation}

\end{quote}
How about covariant vector $L_XY_a$? We write
\begin{equation}
L_XY_a = \lim_{\delta u\to0}\frac{Y_{a}(X+\delta u X) - Y'_{a}(X')}{\delta u}
\end{equation}
Its drag:









\subsubsubsection{Summary} % H4 title

\begin{table}[htbp]
\centering
\begin{tabular}{ll}
\hline
Contravariant & Covariant \\ \hline
$L_X Y^{a} = X^{b}\partial_b Y^{a} - Y^{b}\partial_b X^{a}$ & $L_X Y_{a} = X^{b}\partial_b Y_{a} + Y_{b}\partial_a X^{b}$ \\
\hline
\end{tabular}
\end{table}

\begin{quote}
	Ex: $(1,1)$-type
\begin{equation}
L_{X}T^{a}_{b} = X^{c}\partial_c T^{a}_{b} - T^{c}_{b}\partial_{c}X^{a} + T^{a}_{c}\partial_{b}X^{c}
\end{equation}

\end{quote}
\begin{quote}
	Check for $(2,1)$-type
\end{quote}


\begin{equation}
L_X(Y^{a}Y_{a})=
\end{equation}

\subsubsection{$\odot$ Covariant differentiation} % H3 title

\begin{quote}
	協變微分(引入了connection)
\end{quote}

If we replace the dragged vector by a "\textit{parallel vector}".
\begin{equation}
X^{a}_{\parallel}(x+\delta x) = X^{a}(x) - \delta \bar{X}^{a}
\end{equation}
Notice $\delta \bar{X}^{a}$ very small, so it is propotion to (linear to) $X^{a}$ and $\delta x$, i.e.
\begin{equation}
\delta \bar{X}^{a} = -\Gamma^{a}_{bc} X^{b}\delta x^{c},
\end{equation}
where $\Gamma^{a}_{bc}$ is a 3 indices object connect the ... and ... called "\textit{connection coefficient}".

\begin{quote}
	Notice the repeating indices
\begin{equation}
\delta \bar{X}^{a} = -\sum_{b,c=1}^{n}\Gamma^{a}_{bc} X^{b}\delta x^{c},
\end{equation}

\end{quote}
\underline{Fig}

Then we have
\begin{equation}
\begin{aligned}
\nabla_{c}X^{a} &\equiv \lim_{\delta x^{a}\to 0} \frac{X^{a}(x+\delta x^{a}) - X^{a}_{\parallel}}{\delta x^{a}}\\
&= \lim_{\delta x^{c}\to 0} \frac{\left(X^{a}(x)+\delta x^{c}\partial_c X^{a}\right)-\left(X^{a}(x)-\Gamma^{a}_{bc}X^{v}\delta x^{c}\right)}{\delta x^{c}}
\end{aligned}
\end{equation}
So we define the covariant derivative 
\begin{equation}
\nabla_{c} X^{a} = \partial_{c}X^{a} + \Gamma^{a}_{bc} X^{b}
\end{equation}
We hope $\nabla_c X^{a}$ is a $(1,1)$-type tensor, so we will obtain the transformation rule for connection foefficients $\Gamma^{a}_{bc}$

\begin{quote}
	Check: Is it a $(1,1)$-type tensor?
\begin{equation}
\nabla'_{c}X^{a} = \left(\frac{\partial x'^{a}}{\partial x^{d}}\right)\left(\frac{\partial x^{b}}{\partial x^{c}}\right)\nabla_{b}X^{d}
\end{equation}
Then the left hand side is
\begin{equation}
\begin{aligned}
\mathrm{LHS}
&= \partial'_{c} X'^{a} + \Gamma'^{a}_{bc}X'^{b}\\
&= \partial'_{c}\left(\frac{\partial x'^{a}}{\partial x^{b}}X^{b}\right)
+ \Gamma'^{a}_{bc} \left(\frac{\partial x'^{b}}{\partial x^{d}}X^{d}\right)\\
&= \partial'_{c}\left(\frac{\partial x'^{a}}{\partial x^{b}}\right) X^{b}
+  \frac{\partial x'^{a}}{\partial x^{b}}\partial'_{c}X^{b}
+ \Gamma'^{a}_{bc} \left(\frac{\partial x'^{b}}{\partial x^{d}}X^{d}\right)\\
\end{aligned}
\end{equation}
and the right hand side
\begin{equation}
\begin{aligned}
\mathrm{RHS}
&=\left(\frac{\partial x'^{a}}{\partial x^{d}}\right) \left(\frac{\partial x^{b}}{\partial x'^{c}}\right) \left(\partial_{b}X^{d} + \Gamma^{d}_{eb}X^{e}\right)\\
&=\left(\frac{\partial x'^{a}}{\partial x^{d}}\right) \frac{\partial x^{b}}{\partial x'^{c}} \frac{\partial}{\partial x^{b}}X^{d}
+ \left(\frac{\partial x'^{a}}{\partial x^{d}}\right) \left(\frac{\partial x^{b}}{\partial x'^{c}}\right)\Gamma^{d}_{eb}X^{e}\\
&=\left(\frac{\partial x'^{a}}{\partial x^{d}}\right)\frac{\partial}{\partial x'^{c}}X^{d}
+ \left(\frac{\partial x'^{a}}{\partial x^{d}}\right) \left(\frac{\partial x^{b}}{\partial x'^{c}}\right)\Gamma^{d}_{eb}X^{e}\\
&=\left(\frac{\partial x'^{a}}{\partial x^{d}}\right)\partial'_{c}X^{d}
+ \left(\frac{\partial x'^{a}}{\partial x^{d}}\right) \left(\frac{\partial x^{b}}{\partial x'^{c}}\right)\Gamma^{d}_{eb}X^{e}
\end{aligned}
\end{equation}
Then solving that
\begin{equation}
\begin{aligned}
\mathrm{LHS}
&= \partial'_{c}\left(\frac{\partial x'^{a}}{\partial x^{b}}\right) X^{b}
+  \frac{\partial x'^{a}}{\partial x^{b}}\partial'_{c}X^{b}
+ \Gamma'^{a}_{bc} \frac{\partial x'^{b}}{\partial x^{d}}X^{d}\\
\mathrm{RHS}
&= \frac{\partial x'^{a}}{\partial x^{d}}\partial'_{c}X^{d}
+ \left(\frac{\partial x'^{a}}{\partial x^{d}}\right) \frac{\partial x^{b}}{\partial x'^{c}}\Gamma^{d}_{eb}X^{e}\\
&\Rightarrow
\partial'_{c}\left(\frac{\partial x'^{a}}{\partial x^{b}}\right) X^{b}
+
\Gamma'^{a}_{bc} \frac{\partial x'^{b}}{\partial x^{d}}X^{d}
=
\left(\frac{\partial x'^{a}}{\partial x^{d}}\right) \left(\frac{\partial x^{b}}{\partial x'^{c}}\right)\Gamma^{d}_{eb}X^{e} \\
&\Rightarrow
\left[
\partial'_{c}\left(\frac{\partial x'^{a}}{\partial x^{e}}\right)
+
\Gamma'^{a}_{bc} \frac{\partial x'^{b}}{\partial x^{e}}
\right]X^{e}
=
\left(\frac{\partial x'^{a}}{\partial x^{d}}\right) \left(\frac{\partial x^{b}}{\partial x'^{c}}\right)\Gamma^{d}_{eb}X^{e}
\end{aligned}
\end{equation}
Deriving that
\begin{equation}
\partial'_{c}\left(\frac{\partial x'^{a}}{\partial x^{e}}\right)
+
\Gamma'^{a}_{bc} \frac{\partial x'^{b}}{\partial x^{e}}
=
\frac{\partial x'^{a}}{\partial x^{d}} \frac{\partial x^{b}}{\partial x'^{c}}\Gamma^{d}_{eb}
\end{equation}

\end{quote}
\begin{quote}
	Rmk:
A manifold $M$ with prescribed connection on it is called an "\textit{Affined manifold}" denoted as $(M,\Gamma)$.
\end{quote}

\begin{quote}
	Ex: For the covariant tensor $\nabla_{c} X_{a}$.
Since $\nabla_{c}(X_{a}X^{a}) = \partial_{c}(X_{a}X^{a})$, we can expand
\begin{equation}
\nabla_{c}\left(X_{a}X^{a}\right) = \left(\nabla_{c}X_{a}\right)X^{a} + X_{a}\nabla_{c}X^{a} = \partial_{c}(X_{a}X^{a})
\end{equation}
so that
\begin{equation}
\nabla_{c}X_{a} = \partial_{c} X_{a} - \Gamma^{a}_{bc} X_{b}
\end{equation}

\end{quote}
In general, for the tensor $\nabla_{c}T^{a}_{bc}$


