\subsubsubsection{Tangent space} % H4 title

For a mani fold $M$

\underline{Fig - tangent space}



\begin{quote}
	If $M$ is locally the same as $\mathbb{R}^n$, introducting coordinate patchs
\begin{equation}
\{U_1,U_2,\ldots\} = \text{atlas.}
\end{equation}
In the overlapping $U_\alpha \cap U_\beta \neq \phi$ , $\alpha\neq\beta$
\underline{Fig - overlapping region}
we have $x^{a} = x^{a}(y^1,y^2,\ldots,y^n)$, $a=1,2,\ldots,n$.
\end{quote}

\subsubsection{$\odot$ Tensor transformations of coordinates} % H3 title

Consider a change of coordinates
\begin{equation}
x^{a}\mapsto x'^{a} = f^{a} (x^1,x^2,\ldots,x^n)\equiv x'^{a}(x)
\end{equation}
which means $x'$ is a function of $x$. We define

\begin{quote}
	\textbf{Definition}
A matrix
\begin{equation}
{J'^{a}}_{b} \equiv \frac{\partial x'^{a}}{\partial x^{b}}
\end{equation}
called \textit{Jocobian matrix}, and
\begin{equation}
J'\equiv \abs{J'^{a}_{b}} = \det(J^{a}_{b}).
\end{equation}
Also, by implicit function theorem $x^{a} = x'^{a}(x')$, we have
\begin{equation}
J^{a}_{b} = \frac{\partial x'^{a}}{\partial x^{b}}.
\end{equation}
In fact $J'^{a}_{b}J^{b}_{c} = \delta^{a}_{c}$, then $J' = \det(J^{a}_{b}) = 1/J$. (Notice $\delta^{a}_{b}$ is the Kronecker delta function.)
\end{quote}

\begin{quote}
	Rmk:
\begin{equation}
J'^{a}_{b} = \frac{\partial x'^{a}}{\partial x^{b}}
= \begin{pmatrix}
\frac{\partial x'^{1}}{\partial x^{1}} & \frac{\partial x'^{1}}{\partial x^{2}} & \cdots & \frac{\partial x'^{1}}{\partial x^{n}}\\
\frac{\partial x'^{2}}{\partial x'^{1}} & \frac{\partial x^{2}}{\partial x^{2}} & \cdots & \frac{\partial x'^{2}}{\partial x^{n}}\\
\vdots & \vdots & \ddots & \vdots \\
\frac{\partial x'^{n}}{\partial x^{1}} & \frac{\partial x'^{n}}{\partial x^{2}} & \cdots & \frac{\partial x'^{n}}{\partial x^{n}}\\
\end{pmatrix}
\end{equation}

\end{quote}
\begin{quote}
	Rmk:
$\frac{\partial x^{a}}{\partial x'^{b}}$ can be viewd as coefficients of infinitesmall defferentials, then
\begin{equation}
\begin{aligned}
dx'^{a} &=
\frac{\partial x'^{a}}{\partial x^{1}}dx^{1}
+ \frac{\partial x'^{a}}{\partial x^{2}}dx^{2}
+ \cdots
+ \frac{\partial x'^{a}}{\partial x^{n}}dx^{n} \\
&= \sum_{b=1}^{n} \frac{\partial x^{a}}{\partial x'^{b}} dx^{b}\\
&= \frac{\partial x^{a}}{\partial x'^{b}} dx^{b},\quad \text{Einstein convension}
\end{aligned}
\end{equation}

\end{quote}
We shall classify geometric tensors quantities by transformation properties.

\begin{quote}
	\textbf{Definition}
A \textit{contravariant} vector (rank-1) is a set of quantities $X^{a}$ defined on $p\in M$, such that under coordinate transformation:
\begin{equation}
x^{a}\mapsto x'^{a}(x),
\end{equation}
we have
\begin{equation}
X'^{a} = \frac{\partial x'^{a}}{\partial x^{b}} X^{b}
\end{equation}
\underline{Fig - point p in overlapping patch}
\end{quote}

\begin{quote}
	e.g. tangent vector at $p$ in $M$.
\underline{Fig - tangent vector at p on M}
For $f$ is a function defined on the curve $\gamma$, where
\begin{equation}
f = f(x^{a}(u)),
\end{equation}
and
\begin{equation}
\frac{df}{du} = \frac{\partial f}{\partial x^{a}} \frac{dx^a}{du}
\end{equation}
holds for any $f$. Therefore
\begin{equation}
\frac{d}{du} = \left(\frac{dx^{a}}{du}\right)\frac{\partial}{\partial x^{a}},
\end{equation}
and here where $\partial / \partial x^{a}$​ is like a basis.
Notice that, for $X'^{a} = dx^a/du$
\begin{equation}
X'^{a} = \frac{dx'^a}{du} = \frac{dx'^{a}}{dx^{b}}\frac{dx^{b}}{du} = \frac{dx'^{a}}{dx^{b}}X^{a},
\end{equation}
therefore, $dx^a/du$ is a contravariant vector.
\end{quote}

How about higher rank tensors? e.g. rank-2
\begin{equation}
X'^{ab} = \frac{\partial x^{a}}{\partial x^{c}}\frac{\partial x^{b}}{\partial x^{d}}X^{cd}.
\end{equation}
In general, for rank-$n$ tensor
\begin{equation}
X'^{i_1,\ldots,i_n} =
\frac{\partial x^{i_1}}{\partial x^{j_1}}
\frac{\partial x^{i_2}}{\partial x^{j_2}}
\cdots
\frac{\partial x^{i_3}}{\partial x^{j_3}}
X'^{j_1,\ldots,j_n}
\end{equation}


\begin{quote}
	Rmk:
For a scalar $\phi$ (without indices or rank-0), we have
\begin{equation}
\phi'(x') = \phi(x)
\end{equation}
Here, notice $\phi'$ is not the differentiation of function $\phi$.
\end{quote}

\begin{quote}
	\textbf{Definition}
A covarariant vector (rank-1) is a set of quantities $X_a$ defined at $p\in M$, such that under coordinate transformation, we have
\begin{equation}
X'_{a} = \frac{\partial x^b}{\partial x'^a} X_b.
\end{equation}
e.g. a gradient of $\phi$
\begin{equation}
\frac{\partial \phi}{\partial x'^{a}} = \frac{\partial \phi}{\partial x^{b}} \frac{\partial x^{b}}{\partial x'^{a}},
\end{equation}

\end{quote}
Also, for rank-2 covariant vector 
\begin{equation}
X'_{ab} = \frac{\partial x^{c}}{\partial x^{a}} \frac{\partial x^{d}}{\partial x^{b}} X_{cd}
\end{equation}
If a tensor has a form:
\begin{equation}
X^{a_1,\ldots,a_{p}}_{b_1,\ldots,b_q},
\end{equation}
we called a \textit{mixed-type} tensor denoted by $(p,q)$ type (up,down).

In summary:

\begin{table}[htbp]
\centering
\begin{tabular}{cc}
\hline
tensor & Type \\ \hline
$\phi$ & $(0,0)$ \\
$X^{a}$ & $(1,0)$ \\
$X_{a}$ & $(0,1)$ \\
$X^{ab}$ & $(2,0)$ \\
$X_{ab}$ & $(0,2)$ \\
\hline
\end{tabular}
\end{table}

\subsubsubsection{Coordinate-independent} % H4 title

Physical laws are described by tensor equatios which are coordinates-independent, e.g. Suppose a law is written in the form
\begin{equation}
X_{ab} = Y_{ab}
\end{equation}
in $x$-system, $X$ and $Y$must the same type, since the physical laws must valid in any system. We rewrite the equation to $X_{ab} - Y_{ab} = 0$, and this equation must holds for $X'_{cd} - Y'_{cd} = 0$, and since the transformation is 
\begin{equation}
\begin{aligned}
X'_{cd} = \frac{\partial x^{a}}{\partial x'^{c}} \frac{\partial x^{d}}{\partial x'^{b}} X_{cd}\\
Y'_{cd} = \frac{\partial x^{a}}{\partial x'^{c}} \frac{\partial x^{d}}{\partial x'^{b}} Y_{cd}
\end{aligned}
\end{equation}
so that
\begin{equation}
\left(X'_{ab} - Y'_{ab}\right) = \frac{\partial x^{a}}{\partial x'^{c}} \frac{\partial x^{d}}{\partial x'^{b}}  \left(X_{ab} - Y_{ab}\right) = 0
\end{equation}


\paragraph{example} % H5 title

Maxwell's equations for $\vec{E} = (E_1,E_2,E_3)$ and $\vec{B} = (B_1,B_2,B_3)$ is given by
\begin{equation}
F_{\mu\nu} = - F_{\mu\nu},
\end{equation}
where the tensor is
\begin{equation}
F_{\mu\nu}=
\begin{pmatrix}
0			&E_{1}/c	&E_{2}/c	&E_{3}/c\\
-E_{1}/c	&0			&-B_{3}		&B_{2}\\
-E_{2}/c	&B_{3}		&0			&-B_{1}\\
-E_{3}/c	&-B_{2}		&B_{1}		&0
\end{pmatrix}
\end{equation}
and for $j^{\mu} = \left(\rho c,\vec{J}\right)$, we have
\begin{equation}
\partial_{\mu}F^{\mu\nu} = j^{nu},
\end{equation}
rand-1 controvariant equation under Lorentz transformation $x'^{\mu} = x^{\mu}(x)$​ .

\subsubsection{$\odot$ Tensor fiels} % H3 title

\begin{quote}
	\textbf{Definition}
Over $M$, we asign smoothly every point a tensor, which forms a tensor field.
\end{quote}

e.g. a vector field

\underline{Fig - vector field}

Now, we denote as 
\begin{equation}
X^{ab}(x),\quad x\in M.
\end{equation}


\subsubsection{$\odot$ Elementary operation of tensors} % H3 title

\paragraph{1. Addition} % H5 title

For three $(1,2)$-type tensors $X^{a}_{bc}$, $Y^{a}_{bc}$ and $X^{a}_{bc}$, we have
\begin{equation}
X^{a}_{bc} = Y^{a}_{bc} + Z^{a}_{bc}
\end{equation}


\paragraph{2. scalar multiplication} % H5 title

For $k\in\mathbb{R}$, we have $kX^{a}_{bc}$

\paragraph{3. symmetrization} % H5 title


\begin{equation}
\begin{aligned}
X_{ab} &= \frac{1}{2}\left(X^{ab} + X^{ba}\right) + \frac{1}{2}\left(X^{ab} - X^{ba}\right)\\
&= X_{(a,b)} + X_{[a,b]}
\end{aligned}
\end{equation}

then 
\begin{equation}
\begin{aligned}
X_{(a,b)} &= + X_{(b,a)}\\
X_{[a,b]} &= - X_{[b,a]}
\end{aligned}
\end{equation}


and $X_{(a,b)}$ called symmetric tensor, $X_{[a,b]}$ called anti-symmetric tensor.


