\subsubsubsection{Recall} % H4 title

\begin{itemize}
	\item $(M,\Gamma)$​ affine manifold

\end{itemize}
\begin{itemize}
	\item $\Gamma^{a}_{bc}$ affine connection

\end{itemize}
\begin{itemize}
	\item e.g. for a $(1,1)$-type tensor): $\nabla_{c}X^{a} = \partial_{c}X^{a} + \Gamma^{a}_{bc}X^{b}$

\end{itemize}
\begin{itemize}
	\item trosion free: $\Gamma^{a}_{bc} = \Gamma^{a}_{cb}$

\end{itemize}
\begin{itemize}
	\item Integrable: parrellel transport is path independent

\end{itemize}
\begin{itemize}
	\item affine geodesic $x^{a}(u)$

\begin{equation}
\frac{d^2x^{a}}{du^2} + \Gamma^{a}_{bc}\frac{dx^{b}}{du}\frac{dx^{b}}{du} = 0
\end{equation}

\end{itemize}
\begin{itemize}
	\item Riemannian tensor ($(1,3)$-type tensor)

\begin{equation}
R^{a}_{bcd} =
\partial_{c}\Gamma^{a}_{bd}
- \partial_{d}\Gamma^{a}_{bc}
+ \Gamma^{e}_{bd}\Gamma^{a}_{ec}
- \Gamma^{e}_{bc}\Gamma^{a}_{ed}
\end{equation}

\end{itemize}
\begin{itemize}
	\item How flat?

\end{itemize}
	\begin{itemize}
		\item $M$ is affine flat $\Leftrightarrow$ $\Gamma^{a}_{bc}$​ is tegrable symmetry
		\item $\Gamma^{a}_{bcd} = 0$ $\Leftrightarrow$ $\Gamma^{a}_{bc}$ is tegrable symmetry
		\item $M$ is affine flat $\Leftrightarrow$$\Gamma^{a}_{bcd} = 0$

	\end{itemize}
\begin{itemize}
	\item geodesic coordinate $\{x^{a}\}$: $\Gamma^{a}_{dc} = 0$

\end{itemize}
\subsection{Metric} % H2 title

\begin{quote}
	度規:$(M,g_{\mu\nu})$
\end{quote}

If we add an additional structure called mretric $g_{ab}(x)$, a $(0,2)$-type symmetric tensor.

\subsubsubsection{Def: (Riemannian manifold)} % H4 title

A Riemannian manifold $(M,g)$ where $g_{ab}$ is the metric defoned by
\begin{equation}
(ds)^{2} = g_{ab}dx^{a}dx^{b}
\end{equation}
also called \textit{1st fundamental form}.

\underline{Fig}

\begin{quote}
	e.g. in $\mathbb{R}^{3}$
\begin{equation}
(ds)^2 = (dx)^2 + (dy)^2 + (dz)^2
\end{equation}
where
\begin{equation}
g_{ab}=\begin{pmatrix}
1 & 0 & 0\\
0 & 1 & 0\\
0 & 0 & 1\\
\end{pmatrix}.
\end{equation}

\end{quote}
\begin{quote}
	e.g. in $\mathbb{R}^{2}$
\begin{equation}
(ds)^2 = (dr)^2 + r^2(d\theta)^2 = (dx)^2 + (dy)^2
\end{equation}
where
\begin{equation}
g_{ab}\bigg|_{\rm polar} = \begin{pmatrix}1 & 0\\ 0 & r^2\end{pmatrix}
,\quad
g_{ab}\bigg|_{\rm Cartisian} = \begin{pmatrix}1 & 0\\ 0 & r^2\end{pmatrix}
\end{equation}

\end{quote}
For $x^{a} \in T_{p}(M)$, we define
\begin{equation}
\cos(X,Y) = \frac{g_{ab}X^{a}Y^{b}}{\sqrt{|X|^2|Y|^2}}
\end{equation}
where $|X|^2 = g_{ab}X^{a}X^{b}$. Notice that
\begin{equation}
\begin{cases}
|X|^2 > 0 & \text{positive}\\
|X|^2 < 0 & \text{negative}
\end{cases}
\end{equation}


\begin{quote}
	\textbf{Rmk:}
\begin{enumerate}
	\item $X^{a}\perp Y^{b}$, if $g_{ab}X^{a}Y^b = 0$

	\item $g_{ab}$ is non-singular, if $\det(g_{ab}) \neq 0$

	\item $(g_{ab})^{-1}=g^{ab}$, that is $g_{ab}g^{ab} = \delta^{a}_{b}$

	\item $g_{ab}$ and $g^{ab}$ can be used to lowering and rasing tensorial indices. e.g.

\begin{equation}
\begin{aligned}
X^{a} = g^{ab}X_{b}\\
X_{a} = g_{ab}X^{b}\\
\end{aligned}
\end{equation}
Notice that
\begin{equation}
g_{ab}T^{bc} = {T_{a}}^{c}
\end{equation}
In general,
\begin{equation}
{X_{b}}^{a} = g^{ac}X_{bc}
\quad\neq\quad
{X^{a}}_{b} = g^{ac}X_{cb}
\end{equation}

\end{enumerate}
\subsubsubsection{Geodesic equation} % H4 title

\end{quote}
For the Riemannian manifold $(M,g)$,
\begin{equation}
(ds)^2 = g_{ab} dx^a dx^b
\end{equation}
Then the path $x^{a}(u)$  from $u=P$ to $u=Q$, can be interpreted as
\begin{equation}
S
= \int_{P}^{Q} ds
= \int_{P}^{Q} \sqrt{g_{ab}dx^a dx^b}
= \int_{P}^{Q} \sqrt{g_{ab}\frac{dx^a}{du}\frac{dx^b}{du}} du
\end{equation}
then if we want to minimize the path to find the shortest path from $P$ to $Q$ , we can define the Lagrangian to be
\begin{equation}
L(x^{a}, x^{b}, u) = \sqrt{g_{ab}\frac{dx^a}{du}\frac{dx^b}{du}} = \sqrt{g_{ab}\dot{x}^{a}\dot{x}^b}
\end{equation}
the the path reduced to 
\begin{equation}
S = \int_{P}^{Q}L(x^{a}, x^{b}, u)du
\end{equation}
Using the \textit{Least Action Principle} the shortest path $x^{a}$ corresponding to the solution for the \textit{Euler-Lagrangian equation}
\begin{equation}
\frac{d}{du}\left(\frac{\partial L}{\partial \dot{x}^{a}}\right) = \frac{\partial L}{\partial x^{a}}
\end{equation}
the EL equation becomes
\begin{equation}
g^{ab}\ddot{x}^{b} + \left(\partial_{c}g_{ab} -\frac{1}{2}\partial_{c}g_{db}\right) ..
\end{equation}



choosing a linear parameter $u=\alpha s + \beta$, the equation becomes 
\begin{equation}
\frac{d^2x^a}{ds^2} + \left\{\begin{matrix}a\\bc\end{matrix}\right\} \frac{dx^{b}}{ds}\frac{dx^{c}}{ds} = 0
\end{equation}


\begin{quote}
	\textbf{Recall:}
The affine geodesic
\begin{equation}
\frac{d^2x^a}{ds^2} + \Gamma^{a}_{bc} \frac{dx^{b}}{ds}\frac{dx^{c}}{ds} = 0
\end{equation}


\end{quote}

Comparing to the affine geodesic, define
\begin{equation}
\left\{\begin{matrix}a\\bc\end{matrix}\right\} = \frac{1}{2}g^{ad}\left\{bc,d\right\}
= \frac{1}{2}g^{ad} \left(\partial_{b}g_{dc} + \partial_{c}g_{db} -\partial_{d}g_{bc}\right)
\end{equation}
That is we can deine a \textit{metric connection} $\Gamma$ by the metric $g$ by
\begin{equation}
\Gamma^{a}_{bc} = \frac{1}{2}g^{ad} \left(\partial_{b}g_{dc} + \partial_{c}g_{db} -\partial_{d}g_{bc}\right)
\end{equation}
called \textit{Christoffel} symbols.

\begin{quote}
	HW: prove $\Gamma^{a}_{bc}$ is metric connection $\Leftrightarrow$ $\nabla_{c}g_{ab} = 0$
\end{quote}

\subsubsection{$\odot$ Affine flatness} % H3 title

\subsubsubsection{Def: (metric flatness)} % H4 title

$\exists$ a special coordinate, s.t. $g_{ab}$ is constant everywhere, e.g.
\begin{equation}
g_{ab} = \eta_{ab}
= \begin{pmatrix}
-1 & 0 & 0 & 0\\
0 & 1 & 0 & 0\\
0 & 0 & 1 & 0\\
0 & 0 & 0 & 1\\
\end{pmatrix}
\end{equation}


\subsubsubsection{Theorem: (metrix flatness)} % H4 title

A metric is flat $\Leftrightarrow$ $R^{a}_{bcd} = 0$













That is, once we identify
\begin{equation}
\Gamma^{a}_{bc} = \left\{\begin{pmatrix}a\\bc\end{pmatrix}\right\}
\end{equation}
 then
\begin{equation}
\text{affine flatness $(\Gamma^{a}_{bc}=0)$} = \text{metrix flateness $(g_{ab}=\text{constant})$}
\end{equation}


\subsection{Riemannian tensor} % H2 title

\subsubsubsection{symmetry} % H4 title

\begin{itemize}
	\item $R^{a}_{bcd}$​ has symmeties.
	\item So that $g_{ae}R^{e}_{bcd} = R_{abcd}$ has $4^{4}=256$ components?

\end{itemize}
In fact,
\begin{equation}
\begin{aligned}
R_{abcd} &= - R_{bacd}\quad (a\leftrightarrow b)\\
R_{abcd} &= - R_{abdc}\quad (c\leftrightarrow d)\\
R_{abcd} &= - R_{cdab}\quad (ab\leftrightarrow cd)\\
\end{aligned}
\end{equation}


\subsubsubsection{summation} % H4 title


\begin{equation}
R_{abcd} + R_{adbc} + R_{acbd} = 0
\end{equation}

we can check that in the geodesic equation.

\begin{quote}
	\textbf{Ricci tensor}
\begin{equation}
R_{ab} = R^{c}_{acb}
\end{equation}


\end{quote}

\begin{quote}
	\textbf{Ricci scalar}
\begin{equation}
g^{ab}R_{ab} = {R^{a}}_{a} = R
\end{equation}

\end{quote}
\begin{quote}
	\textbf{Einstein equations}
\begin{itemize}
	\item vacuum equation

\begin{equation}
R_{ab} - \frac{1}{2} g_{ab}R = 0
\end{equation}


	\item with matter

\begin{equation}
R_{ab} - \frac{1}{2} g_{ab}R = \frac{8\pi G}{c^4} T_{ab}
\end{equation}
where $T_{ab}$ is energy-momentum tensor.
\end{itemize}

\end{quote}
