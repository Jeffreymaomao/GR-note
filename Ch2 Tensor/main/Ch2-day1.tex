\section{Tensors} % H1 title

\begin{quote}
	Tensors is \textit{generalization of vector}.
\end{quote}

\begin{itemize}
	\item What;s tensors
	\item Tensors Algebra
	\item Tensor differentiation
	\item Tensor integration

\end{itemize}
\subsection{Introduction} % H2 title

\begin{itemize}
	\item \textbf{Newtonian mechanics} $\Rightarrow$ ODE
	\item \textbf{Quantum physics} $\Rightarrow$ Linear Algebra
	\item \textbf{Relativity} $\Rightarrow$ Tensor Calculus

\end{itemize}
\subsubsection{Vector} % H3 title

\begin{quote}
	Ex.
\underline{Ch2 - Fig2}
$\vec{A}$ is a vector
\begin{equation}
\vec{A} = A_1\hat{e}_1 + A_2 \hat{e}_2 = A'_1\hat{e}'_1 + A'_2 \hat{e}'_2
\end{equation}
and
\begin{equation}
\begin{pmatrix}
A'_1\\A'_2
\end{pmatrix}
=
R(\theta)
\begin{pmatrix}
A_1\\A_2
\end{pmatrix}
\end{equation}

\end{quote}
\begin{quote}
	\textbf{Definition}
A vector $\vec{A}$ is an object that tranforms as the same as $\vec{r}$​ under axis rotation.
\end{quote}

In matrix notation:
\begin{equation}
A_i' = \sum_{j=1}^{n}R_{ij}(\theta) A_j,\quad \text{in $\mathbb{R}^n$}
\end{equation}
We denote as 
\begin{equation}
A'_i = R_{ij}A_j,
\end{equation}
also called Einstein's convention (\textit{Summing over repested indeces.})

\subsubsection{Tensor} % H3 title

\textbf{Definition}

A rank-2 tensor $A_{ij}$ is an object such that each index transforms like a vector.

i.e.
\begin{equation}
A'_{ij} = R_{ik}R_{g\ell}A_{k\ell}
\end{equation}


In general, a rank-$n$ tensor $A_{i_1,\ldots,i_n}$ transformation as 
\begin{equation}
A'_{i_1,\ldots,i_n} = R_{i_1,j_1}R_{i_2,j_2}\cdots R_{i_n,j_n}\cdot A_{j_1,\ldots,j_n}
\end{equation}


\begin{quote}
	Rmk:
\begin{itemize}
	\item In $\mathbb{R}^{n}$, $\{\hat{e}_1,\hat{e}_2,\ldots,\hat{e}_n\}$ are constant basis.

	\item In polar coordinate system, basis are not constant

\underline{Ch1 - Fig 2 polar}
\end{itemize}

\end{quote}
\subsection{Manifold and coordinate} % H2 title

\begin{quote}
	Manifold (mfld) 流形
\end{quote}

What is the dimension of a manifold $M$ ?

\underline{Fig3 - A manifold locally likes $R^n$}

\textbf{Definition}

An $n$-dimension manifold $M$ is something which locally look like $\mathbb{R}^n$.
\begin{equation}
\begin{aligned}
x^{i} :
&U \to \mathbb{R}^n\\
&p \mapsto  x(p)
\end{aligned}
\end{equation}


\begin{quote}
	Ex.
$M = S^2$ (Two sphere)
\underline{Fig4 - sphere}
\end{quote}

\begin{quote}
	Rmk.
If $(x^{1},x^{2},\ldots,x^{n})$ be a set of $n$ coordinates , which is non-degenerate if
\begin{equation}
p \longleftrightarrow (x^{1},x^{2},\ldots,x^{n})
\end{equation}
is 1-1 or \textit{one-to-one mapping}.
e.g.
\underline{Fig4 - degenerate sphere}
\end{quote}

In overlapping region

\underline{Fig5 - overlapping}

\begin{quote}
	Ex. \textbf{stereo graphic projection}
spherical coordinate $\theta,\varphi$, notice that it is singular at $\theta = 0$ (north pole) and $\theta = \pi$ (south pole).
\begin{equation}
x^2+y^2+z^2=1
\end{equation}
\underline{Fig6 - projection}
\begin{itemize}
	\item $U_N$:

\begin{equation}
\begin{aligned}
&(x,y,z-1) = \lambda (X,Y,-1)\\
&\Rightarrow \frac{x}{X} = \frac{y}{Y} = \frac{z-1}{-1} = \lambda\\
&\Rightarrow \begin{cases}
\displaystyle X= \frac{x}{\lambda} = \frac{x}{1-z}\\
\displaystyle Y= \frac{y}{\lambda} = \frac{y}{1-z}
\end{cases}
\end{aligned}
\end{equation}

	\item $U_S$:

\begin{equation}
\begin{aligned}
&(x,y,z-1) = \lambda (X,Y,+1)\\
&\Rightarrow \frac{x}{X} = \frac{y}{Y} = \frac{z+1}{+1} = \lambda\\
&\Rightarrow \begin{cases}
\displaystyle X= \frac{x}{\lambda} = \frac{x}{1+z}\\
\displaystyle Y= \frac{y}{\lambda} = \frac{y}{1+z}
\end{cases}
\end{aligned}
\end{equation}

Then, the manifold $M = U_{N}\cup U_{S}$, where $U_{N}\cap U_{S}=\phi$. And we called
	\item Two subset $U_N$ and $U_S$ to be a \textit{patch} of covering (覆蓋).
	\item The set $\{U_N,U_S\}$ is the Altas.



\textbf{Claim}
For a point
\begin{equation}
p\in U_{N}\cap U_S
\end{equation}
we have
\begin{equation}
\begin{cases}
\displaystyle X' = \frac{X}{X^2+Y^2}\\
\displaystyle Y' = \frac{Y}{X^2+Y^2}
\end{cases}
\end{equation}
coordinate transformation between 2 system.
\end{itemize}

\end{quote}
\subsubsubsection{Tensors on $M$} % H4 title

The tensors on $M$ are geometric quantities that obey coordinate transformation in overlapping region.

\subsubsubsection{Curve on $M$} % H4 title

\underline{Fig7 - curves on M}

In general, an $m$-dimension object in $M$ can be parametrized by
\begin{equation}
x^{a} = x^{a}\left(u^1,u^2,\ldots,u^m\right), \quad a = 1,2,\ldots,n
\end{equation}


\begin{quote}
	In particular, if $m=n-1$​, i.e.
\begin{equation}
x^{a} = x^{a} \left(u^1,u^2,\ldots,u^{n-1}\right), \quad a = 1,2,\ldots,n
\end{equation}
is a \textit{hypersurface}. We also using a function $f(x^{1},x^{2},\ldots,x^{n})=0$.
\end{quote}

