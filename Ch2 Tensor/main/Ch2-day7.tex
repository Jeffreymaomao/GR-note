\begin{quote}
	Note:
If it is not torsion-free
\begin{equation}
\left(\nabla_{c}\nabla_{d} - \nabla_{d}\nabla_{c}\right)X^{a}
=  R^{a}_{bcd} X^{b} \left(\partial_{c}\partial_{d}
+ \left(\Gamma^{b}_{cd}-\Gamma^{b}_{dc}\right)\nabla_{b} X^{a}
- \partial_{d}\partial_{c}\right)X^{a} ,
\end{equation}

\end{quote}
\subsubsection{Geodesic coordiante} % H3 title

Locally, we may choose $\{x^{a}\}$ s.t. $\Gamma^{a}_{bc} \stackrel{*}{=} 0$ , $[x^{a}]_{p} \stackrel{*}{=} 0$

Suppose, 
\begin{equation}
x^{a} \mapsto x'^{a} + \frac{1}{2}Q^{a}_{bc} x^{b} x^{c},
\end{equation}
where we require $Q^{a}_{bc} = Q^{a}_{bc}$ (constant), then we calculte
\begin{equation}
\frac{\partial x'^{a}}{\partial x^{d}}\Bigg|_{p} = \delta^{a}_{d} \Bigg|_{p} + Q^{a}_{bc} x^{b}\Bigg|_{p} = \delta^{a}_{d}
\end{equation}
then
\begin{equation}
\frac{\partial^2 x'^{a}}{\partial x^{d}\partial x^{c}}\Bigg|_{p} = Q^{a}_{cd}
\end{equation}


\begin{quote}
	\textbf{Recall:}
\begin{equation}
\begin{aligned}
\Gamma'^{a}_{bc}
&= \left(\frac{\partial x'^{a}}{\partial x^{d}}
\frac{\partial x^{f}}{\partial x'^{c}}
\frac{\partial x^{e}}{\partial x'^{b}}\right)\Bigg|_{p}
\Gamma^{d}_{ef}\Bigg|_{p}
- \frac{\partial^2 x'^{a}}{\partial x^{d}\partial x^{c}}
\frac{\partial x^{d}}{\partial x'^{c}}
\frac{\partial x^{e}}{\partial x'^{b}}\Bigg|_{p}\\
&= \delta^{a}_{b} \, \delta^{f}_{c} \, \delta^{e}_{b} \,\Gamma^{d}_{ef}\Bigg|_{p}
- Q^{a}_{ed}\delta^{d}_{c}\delta^{e}_{b}\\
&= \Gamma^{a}_{bc}\Bigg|_{p} - Q^{a}_{bc}
\end{aligned}
\end{equation}

\end{quote}
So, if we choose 
\begin{equation}
Q^{a}_{bc} = \Gamma^{a}_{bc}\Bigg|_{p}
\end{equation}
then 
\begin{equation}
\Gamma'^{a}_{bc} \stackrel{*}{=} 0
\end{equation}
This "trick" for computation set $\Gamma^{a}_{bc}\Bigg|_{p} = 0$. After computation, restore $\Gamma$ by $\partial_{a}\to \nabla_{a}$.

\begin{quote}
	\textbf{Rmk:}
In general, it is impossible to find a coordinate transforamtion, s.t. $\Gamma^{a}_{bc} = 0$ globally. If it does, then the manifold is affine flat manifold.
\end{quote}

\subsection{Affine flatness} % H2 title

\subsubsection{$\odot$ Integrable connetion} % H3 title

\subsubsubsection{\textbf{Definition}:} % H4 title

If a parallel transport of a vector from $P$ to $Q$ is \textit{independent} pf path then the connection is \textit{integrable}.

\subsubsubsection{Lemma 1.} % H4 title

The connection is \textit{integrable} or \textit{torsion free}t $\Longleftrightarrow$  $R^{a}_{bcd} = 0$.

\paragraph{proof $\Rightarrow$ (necessary)} % H5 title

If $\Gamma^{a}_{bc}$ is integrable $\Rightarrow$ $\displaystyle \frac{dc^{c}}{du} \nabla_{c} X^{a} = 0$ is path integrable $\Rightarrow$ Since, $\displaystyle \frac{dc^{c}}{du}$ is arbitrary. $\nabla_cX^{a}=0$ , that is 
\begin{equation}
\nabla_{c} X^{a} = \partial_{c} X^{a} + \Gamma^{a}_{bc} X^{b} = 0
\end{equation}
which is 1st order P.D.E. the existence of solution: $\partial_{d}\partial_{c}X^{a} = \partial_{c}\partial_{d}X^{a}$, 
\begin{equation}
\begin{aligned}
\nabla_{c}\nabla_{d} X^{a} - \nabla_{d}\nabla_{c} X^{a}
&= R^{a}_{bcd} X^{b} + \left(\Gamma^{e}_{cd} - \Gamma^{e}_{dc}\right)\nabla_{e}X^{a} + \left(\partial_{d}\partial_{c}X^{a} - \partial_{c}\partial_{d}\right)X^{a}\\
&= R^{a}_{bcd} X^{b} = 0
\end{aligned}
\end{equation}
Since $X^{b}$ is arbitrary, $R^{a}_{bcd} = 0$.

\paragraph{proof $\Leftarrow$ (sufficient)} % H5 title

Consider an infinitesmal loop:

\underline{FIg}

compute parallel transport along two paths.

\begin{enumerate}
	\item For the path $C_1$:

\end{enumerate}
	\begin{itemize}
		\item $x^{a}\to x^{a} + \delta x^{a}$

\begin{equation}
\begin{aligned}
X^{a}(x+\delta x)
&= X^{a}(x) + \bar{\delta} X^{a}(x)\\
&= X^{a}(x) - \Gamma^{a}_{bc} X^{b}\delta x^{c}
\end{aligned}
\end{equation}

	\end{itemize}
	\begin{itemize}
		\item $x^{a}+\delta x^{a}\to (x^{a} + \delta x^{a}) + dx^{a}$

\begin{equation}
\begin{aligned}
X^{a}(x+\delta x+dx)
&= X^{a}(x+\delta x) + \bar{\delta} X^{a}(x+\delta x)\\
&= \left(X^{a}(x) - \Gamma^{a}_{bc} X^{b}\delta x^{c}\right)
- \left(\Gamma^{a}_{bc}+\partial_{d}\Gamma^{a}_{bc}\delta x^{d}\right)
\left(x^{b} - \Gamma^{b}_{ef}X^{e}\delta x^{f}\right) dx^{c}
\end{aligned}
\end{equation}

	\end{itemize}

\begin{enumerate}
	\item For the path $C_2$:

\begin{equation}
\begin{aligned}
X^{a}(x+dx+\delta x)
&= X^{a} - \Gamma^{a}_{bc} X^{b} dx^{c} - \Gamma^{a}_{bc} x^{b}\delta x^{c}\\
&=
\end{aligned}
\end{equation}

\end{enumerate}

Last we have the difference 
\begin{equation}
\begin{aligned}
\Delta X &= X^{a}_{C_1}(x+\delta x+dx) - X^{a}_{C_2}(x+\delta x+dx)\\
&= \left(\partial \Gamma^{a}_{bd} - \partial\right) X^{b}\delta x^{d}\delta x^{c}\\
&= R^{a}_{cbd}
\end{aligned}
\end{equation}


\subsubsubsection{Lemma 2} % H4 title

A manifold $M$ is affine flat ($\Gamma^{a}_{bc}=0$ globally) $\Longleftrightarrow$ The connection symmetric and integrable.

\paragraph{proof $\Rightarrow$ (necessary)} % H5 title

If $M$ is affoine flat, then $\Gamma^{a}_{bc}=0$ everywhere, then parallel transport is path independent, trivially.

\paragraph{proof $\Leftarrow$ (sufficient)} % H5 title

If $\Gamma^{a}_{bc}$ is integrable, around $P$ choose L.T. vector  $\{X^{a}_{1},\ldots,X^{a}_{n}\}$, where $\operatorname{dim} M = n$. Now, using $\Gamma^{a}_{bc}$ to parallel transport $\{X^{a}_{i}\}$ everywhere. 

Hence, for any $x\in M$, $\{X^{a}_{i}(x)\}$ is L.T., then $\abs{X^{a}_{i}}\neq 0$ (by L.T.), so $\exists !$ inverse $X^{i}_{b}$ s.t. 
\begin{equation}
X^{a}_{i}X^{i}_{a} = \delta^{a}_{b}.
\end{equation}
Since
\begin{equation}
0 = \nabla_{b} X^{a}_{i} = \partial_{b} X^{a}_{i} + \Gamma^{a}_{cb} X^{c}_{i}
\quad\Rightarrow\quad
\Gamma^{a}_{cb} = - X^{i}_{c} \partial X^{a}_{i}.
\end{equation}
Thus 
\begin{equation}
o = \Gamma^{a}_{bc} - \Gamma^{a}_{cb} = X^{i}_{c}\partial_{b} X^{a}_{i} - X^{i}_{b} \partial_{c}X^{a}_{i} = \left(X^{i}_{c}\partial_{b} - X^{i}_{b} \partial_{c}\right)X^{a}_{i}
\end{equation}
Because, $(X^{a}_{i})$ is non-degenerate $\Rightarrow$ $\partial_c X^{i}_{b} = \partial_{b} X^{i}_{c}$, then $\forall x\in M, \exists$ functions $f^{i}(x)$ s.t. $X^{i}_{b} = \partial_{b} f^{i}$

