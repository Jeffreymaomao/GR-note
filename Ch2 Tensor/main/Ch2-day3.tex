\begin{quote}
	For rank-$n$, $n\geq 3$, e.x
\begin{equation}
X^{a_1,a_2,\ldots,a_n},\quad a_1,a_2,\ldots,a_n = 1,2,\ldots,n
\end{equation}
we have $n^3$ elements. We define
\begin{equation}
\begin{aligned}
X_{(a_1,a_2,\ldots,a_N)} &= \frac{1}{n!}\sum \text{all permutations}\\
X_{[a_1,a_2,\ldots,a_N]} &= \frac{1}{n!}\sum \text{alternating permutations}
\end{aligned}
\end{equation}


Ex: $n=3$, for anti-symmetric tensor
\begin{equation}
\begin{aligned}
X_{[a,b,c]} = \frac{1}{6} \left(
X_{abc}+X_{bca}+X_{cab}
-X_{bac}-X_{cba}-X_{acb}\right)
\end{aligned}
\end{equation}
For aymmetric tensor
\begin{equation}
X_{(a,b,c)} = \frac{1}{6}\left(++++++\right)
\end{equation}
\textbf{Check}: Also, the symmetrized tensor still remain the properties of original tensor, e.x.
\begin{equation}
X'_{(a,b)} = \frac{\partial x^{c}}{\partial x'^{a}}\frac{\partial x^d}{\partial x'^{b}}X_{(c,d)}
\end{equation}

\end{quote}
\paragraph{4. tensor product} % H5 title

Ex: For $(1,1)$-type tensor $Y^{a}_{b}$ and $(0,2)$-type tensor $Z_{cd}$, we want to construct a $(1,3)$-type tensor. 

\begin{quote}
	\textbf{Define}
\begin{equation}
X^{a}_{bcd} = Y^{a}_{b} Z_{cd}
\end{equation}
Check:
\begin{enumerate}
	\item $\displaystyle Y'^{a}_{b} = \frac{\partial x'^{a}}{\partial x^{e}}\frac{\partial x'^{f}}{\partial x^{b}}Y^{e}_{f}$
	\item $\displaystyle Z'_{cd} = \frac{\partial x^{g}}{\partial x'^{c}}\frac{\partial x^{h}}{\partial x'^{d}}Z^{gh}$

then
\begin{equation}
\begin{aligned}
X'^{a}_{bcd} &= Y'^{a}_{b} Z'_{cd}\\
&= \frac{\partial x'^{a}}{\partial x^{e}}\frac{\partial x'^{f}}{\partial x^{b}}\frac{\partial x^{g}}{\partial x'^{c}}\frac{\partial x^{h}}{\partial x'^{d}}Y^{e}_{f}Z^{gh}
\end{aligned}
\end{equation}

\end{enumerate}
\paragraph{5. tensor contraction} % H5 title

\end{quote}
\begin{quote}
	\textit{contraction} means summing over some indices
\end{quote}


\begin{equation}
X^{a}_{bcd} \quad\longrightarrow\quad X^{c}_{bcd}
\end{equation}

that is $(1,3)$-type to $(0,2)$-type. That is 
\begin{equation}
X^{c}_{acd} = X^{1}_{a1d} + X^{2}_{a2d} + \cdots + X^{n}_{and}
\end{equation}


\begin{quote}
	Notice that repeated need to be sum, so it is not a $(1,3)$-type anymore.
\end{quote}

\begin{quote}
	Check:
\begin{equation}
\begin{aligned}
X'^{c}_{bcd}
&= \frac{\partial x'^{c}}{\partial x^{e}}
\frac{\partial x^{f}}{\partial x'^{b}}
\frac{\partial x^{g}}{\partial x'^{c}}
\frac{\partial x^{h}}{\partial x'^{d}} X^{e}_{fgh}\\
&= \left(\frac{\partial x'^{c}}{\partial x^{e}}
\frac{\partial x^{g}}{\partial x'^{c}}\right)
\frac{\partial x^{f}}{\partial x'^{b}}
\frac{\partial x^{h}}{\partial x'^{d}} X^{e}_{fgh}\\
&= \delta^{g}_{e}
\frac{\partial x^{f}}{\partial x'^{b}}
\frac{\partial x^{h}}{\partial x'^{d}} X^{e}_{fgh}\\
&= \frac{\partial x^{f}}{\partial x'^{b}}
\frac{\partial x^{h}}{\partial x'^{d}} X^{g}_{fgh}\\
\end{aligned}
\end{equation}

\end{quote}
\subsubsection{Vector fields} % H3 title

\underline{Fig vector filed}

\begin{quote}
	\textbf{Define}
A vector filed on $M$ is an assignment (smoothly) of tangent at each point of $M$.
\end{quote}

Recall:

\underline{Fig cirlce}

In patch $\{x^{a}\}$, the tangent vector in this patch is expressed as 
\begin{equation}
X = X^{a}\frac{\partial}{\partial x^{a}}
= \underbrace{X^{a}}_{\rm component} \underbrace{\frac{\partial}{\partial x^{a}}}_{\rm basis}
= X'^{b}\frac{\partial}{\partial x'^{b}}
\end{equation}


\begin{quote}
	Similar to $\displaystyle \vec{A} = \sum_{i=1}^{3}a_{i}\hat{e}_i = \sum_{j=1}^{3}a'_{j}\hat{e}'_j$
\end{quote}

\begin{quote}
	we then check
\begin{equation}
\frac{\partial}{\partial x^{a}} = \frac{\partial x'^{b}}{\partial x^{a}} \frac{\partial }{\partial x'^{b}}
\end{equation}
so
\begin{equation}
X
= X^{a}\frac{\partial}{\partial x^{a}}
= X^{a}\frac{\partial x'^{b}}{\partial x^{a}} \frac{\partial }{\partial x'^{b}}
= X'^{b} \frac{\partial }{\partial x'^{b}}
\end{equation}
so the component is a controvariant.
\end{quote}



We need to clearify the notation. $T_{p}(M)$ is the tangent space at $p$, then 
\begin{equation}
T(M)=\bigcup_{p} T_{p}(M)
\end{equation}
is a vector field, where $X\in T(M)$ and $X\bigg|_{p} = T_{p}(M)$

\begin{quote}
	\textbf{Lie bracket}
Suppose $X$ and $Y$ are two vector field, i.e. locally
\begin{equation}
X=X^{a}\frac{\partial}{\partial x^{a}}
\quad\text{and}\quad
Y=Y^{b}\frac{\partial}{\partial x^{b}}.
\end{equation}
Then the \textit{Lie bracket} is still a vector field.
\textbf{Check}: For a function $f$
\begin{equation}
\begin{aligned}
\,[X,Y] f
&= \left(XY-YX\right)f\\
&= \left(XY^{b}\frac{\partial}{\partial x^{b}}-YX^{a}\frac{\partial}{\partial x^{a}}\right)f\\
&= \left(
X^{a} \frac{\partial}{\partial x^{a}}
\left(Y^{b}\frac{\partial f}{\partial x^{b}}\right)
-
Y^{b}\frac{\partial}{\partial x^{b}}
\left(X^{a}\frac{\partial f}{\partial x^{a}}\right)\right)\\
&= X^{a}\left(\frac{\partial Y^{b} }{\partial x^{a}}\frac{\partial f}{\partial x^{b}} + Y^{b}\frac{\partial^2 f}{\partial x^{a}x^{b}}\right)
-
Y^{b}\left(\frac{\partial X^{a}}{\partial x^{b}}\frac{\partial f}{\partial x^{a}}+X^{a}\frac{\partial^2 f}{\partial x^{b}x^{a}}\right)\\
&= X^{a}\frac{\partial Y^{b} }{\partial x^{a}}\frac{\partial f}{\partial x^{b}}
+ X^{a}Y^{b}\frac{\partial^2 f}{\partial x^{a}x^{b}}
- Y^{b}\frac{\partial X^{a}}{\partial x^{b}}\frac{\partial f}{\partial x^{a}}
+Y^{b}X^{a}\frac{\partial^2 f}{\partial x^{b}x^{a}}\\
&= X^{a}\frac{\partial Y^{b} }{\partial x^{a}}\frac{\partial f}{\partial x^{b}}
- Y^{b}\frac{\partial X^{a}}{\partial x^{b}}\frac{\partial f}{\partial x^{a}}\\
&= \left(X^{a}\frac{\partial Y^{b} }{\partial x^{a}}\frac{\partial}{\partial x^{b}} - Y^{b}\frac{\partial X^{a}}{\partial x^{b}}\frac{\partial}{\partial x^{a}}\right)f
\end{aligned}
\end{equation}
Now we define the Lie bracket to be $\displaystyle Z = [X,Y] = Z^{a}\frac{\partial}{\partial x^{a}}$, since
\begin{equation}
Z=[X,Y] = X^{b}\frac{\partial Y^{a} }{\partial x^{b}}\frac{\partial}{\partial x^{a}} - Y^{b}\frac{\partial X^{a}}{\partial x^{b}}\frac{\partial}{\partial x^{a}}
= \left(X^{b}\frac{\partial Y^{a} }{\partial x^{b}} - Y^{b}\frac{\partial X^{a}}{\partial x^{b}}\right)\frac{\partial}{\partial x^{a}}
= Z^{a}\frac{\partial}{\partial x^{a}}
\end{equation}
Let
\begin{equation}
Z^{a} =X^{b}\frac{\partial Y^{a} }{\partial x^{b}} - Y^{b}\frac{\partial X^{a}}{\partial x^{b}}
\end{equation}
\textbf{Check}: $Z^{a}$ is a contravariant vector.


\end{quote}

