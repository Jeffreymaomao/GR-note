
% Sets the document class and font size
\documentclass[12pt]{article}
\usepackage[a4paper, margin=1in]{geometry}

\usepackage[utf8]{inputenc}    % Input encoding
\usepackage[T1]{fontenc}       % Font encoding
\usepackage{fontspec}          % font encoding

% Advanced math typesetting
\usepackage{amsmath}
\usepackage{amssymb}
\usepackage{mathtools}
\usepackage{physics}

% Symbols and Text
\usepackage{bbold}     % bold font
\usepackage{ulem}      % strikethrough
\usepackage{listings}  % Source code listing
\usepackage{import}    % Importing code and other documents

% graphics
\usepackage[dvipsnames]{xcolor}
\usepackage{graphicx}
\usepackage{tikz}
\usepackage{pgfplots}
\usepackage{tcolorbox}

% figure
\usepackage{float}
\usepackage{subfigure}

\usepackage{hyperref}  % Hyperlinks in the document
\hypersetup{
    colorlinks=true,
    linkcolor=Blue,
    filecolor=red,
    urlcolor=Blue,
    citecolor=blue,
    pdftitle={Article},
    pdfauthor={Author},
}

\usepackage{xeCJK}             % Chinese, Japanese, and Korean characters
\setCJKfamilyfont{kai}{標楷體}
%================================================================================================
\begin{document}

\section{Ch.1 Special Relativity (Review)} % H1 title

space time $(t,\vec{x})\to$ event 

for (1+1)-dim space time

\underline{Fig1 - world line}

An observer : with a clock and a ruler

Constitutes and inertial frame

\subsection{Newtonian theory} % H2 title

\begin{itemize}
	\item two observer:  $(t,x,y,z)$ and $(t',x',y',z')$

\end{itemize}
\begin{itemize}
	\item relative speed $v$ in $x$-direction

\end{itemize}
\subsubsection{Galilean transformation} % H3 title


\begin{equation}
\left\{\begin{aligned}
x &= x' + vt\\
y &= y'\\
z &= z'\\
t &= t'\quad \text{(time is absolute)}\\
\end{aligned}\right.
\end{equation}

\subsubsection{Principle of equivalence in Newton's theory} % H3 title

\begin{enumerate}
	\item For the event in $S$ and $S'$, time is absolute
    $$    \begin{aligned}    S  &: (t,x,y,z)\\    S' &: (t',x',y',z')    \end{aligned}    $$    If we want to measure the length of a moving object

\end{enumerate}
\begin{quote}
	We need to measure the coordinates of ends simutaneously.
\underline{Fig2 - measure ends}
\end{quote}

\begin{enumerate}
	\item speed

\end{enumerate}
    $$    \begin{cases}    \dot{x} = \dot{x}' + v\\    \dot{y} = \dot{y}'\\    \dot{z} = \dot{z}'\\    \end{cases}    $$

\begin{enumerate}
	\item accerleration

\end{enumerate}
    $$    \ddot{x} = \ddot{x}',    \quad \ddot{y} = \ddot{y}',    \quad \ddot{z} = \ddot{z}'    $$

\subsection{Special Relativity} % H2 title

\begin{quote}
	Einstein: Two postitulates of S.R.
P1. All inertial observers are eqivalent.
P2. The speed of light of light $c$ is the smae for all observers ($c=299792458\,\mathrm{m/s}$).
\end{quote}

If we set $c=1$ (relativistic unit)

\underline{Fig 3 - xt diagram}

\subsubsection{$\odot$ Lorentz transformation (\textit{Bondi} $k$-factor)} % H3 title

\begin{quote}
	$k$ factor characterize the difference between Newton's theory and S.R.
\end{quote}

\underline{Fig4 - xt diagram with AB}

so that $k=k(v)$, and it must be a linear relation between $(t,x)$ and $(t',x')$ since the free motion must be the same, i.e.
\begin{equation}
\text{free motion:}\quad
\begin{cases}
x = x_0 + u t\\
x' = x_0' + u' t'
\end{cases}
\end{equation}
\underline{Fig5 - triangle} 

then if $t_1=T$ and $t_2=k^2T$ (see Fig5 - t1t2), we have
\begin{equation}
\begin{cases}
t = (k^2+1)T/2\\
x = (k^2-1)T/2\\
\end{cases}
\end{equation}
since $v=x/t=(k^2-1)/(k^2+1)<1$ we have the factor
\begin{equation}
k=\sqrt{\frac{1+v}{1-v}}
\end{equation}


\begin{quote}
	Rmk (Relativistic Doppler effect):
The frequence $\omega = 2\pi/T$, we have
\begin{equation}
\begin{aligned}
T\to T'=kT\\
\omega \to \omega' = \omega/k
\end{aligned}
\end{equation}
If $v>0\Rightarrow k>1$: $\omega'<\omega$ (red shift)
If $v<0\Rightarrow k<1$: $\omega'<\omega$ (blue shift)
\end{quote}

In Newton's theory 
\begin{equation}
\begin{cases}
x=x'+vt\\
\dot{x} = \dot{x}' + v\quad\text{(Additoin formula)}
\end{cases}
\end{equation}
\underline{Fig6 - ABC world line}
\begin{equation}
\begin{aligned}
k_{AB} &= \sqrt{\frac{1+v_{AB}}{1-v_{AB}}}\\
k_{BC} &= \sqrt{\frac{1+v_{BC}}{1-v_{BC}}}\\
k_{AC} &= \sqrt{\frac{1+v_{AC}}{1-v_{AC}}}
= \sqrt{\frac{(1+v_{AB})(1+v_{BC})}{(1-v_{AB})(1-v_{BC})}}\\
\end{aligned}
\end{equation}
solving that 
\begin{equation}
v_{AC} = \frac{v_{AB}+v_{BC}}{1+v_{AB}v_{BC}} = \frac{v_{AB}+v_{BC}}{1+v_{AB}v_{BC}/c^2}
\end{equation}


\begin{quote}
	example:
if $v_{BC} = c$​, we have $\displaystyle v_{AC}=\frac{c+v_{BC}}{1+v_{BC}c/c^2} = c$
if $v_{AC} = c$, we have $\displaystyle v_{AC}=\frac{v_{AC}+c}{1+cv_{BC}/c^2} = c$
\end{quote}

In fact,
\begin{equation}
v_{AV}-1 = \frac{(v_{AC}-1)(1-v_{BC})}{1+v_{AB}v_{BC}} < 0
\end{equation}
when $v_{AB},v_{BC}<1$: S.R. $\to$ Newton

\begin{quote}
	Rmk:
\begin{enumerate}
	\item massive particles $m\neq 0$: $v<1$ or $v<c$

	\item Massless particle $m=1$: $v=1$ or $v=c$

e.g. \textit{photon}, \textit{graviton}, ~~\textit{neutrino}~~
\end{enumerate}

\end{quote}
What about the Relation between two space time coordinate of the same event $P$:

\underline{Fig7 lorentz - transformation}

by $k$-factor

\begin{enumerate}
	\item $t'-x'=k(t-x)$
	\item $t+x=k(t'+x')$

\end{enumerate}
solving that
\begin{equation}
\begin{aligned}
t' = \frac{t-vx}{\sqrt{1-v^2}}\\
x' = \frac{x-vt}{\sqrt{1-v^2}}
\end{aligned},\quad\text{where $(c=1)$}.
\end{equation}


\begin{quote}
	Rmk:
\begin{enumerate}
	\item $t^2-x^2 = (t')^2 - (x')^2$ (Nom-Euclidean) It's called \textit{Minkkowski space}.

	\item comparision

\begin{table}[htbp]
\centering
\begin{tabular}{ll}
\hline
Galilean & Lorentz \\ \hline
$t'=t$ & $t'=\frac{t-vx}{\sqrt{1-v^2}}$ \\
$x'=x-vt$ & $x'=\frac{x-vt}{\sqrt{1-v^2}}$ \\
$y'=y$ & $y'=y$ \\
$z'=z$ & $z'= z$ \\
\hline
\end{tabular}
\end{table}
\underline{Fig8 -moving frame}
	\item \textbf{Definition}: interval between two event $P_1$ and $P_2$:


\end{enumerate}
\begin{itemize}
	\item fiinite Interval


\begin{equation}
s^2 = (t_2-t_1)^2 - (x_2-x_1)^2 - (y_2-y_1)^2 - (z_2-z_1)^2
\end{equation}

	\item infenitestmall interval

\begin{equation}
(ds)^2 = (dt)^2 - (dx)^2 - (dy)^2 - (dz)^2
\end{equation}


\end{itemize}
\begin{enumerate}
	\item revise the order

\begin{equation}
\begin{aligned}
t' = \frac{t-vx/c^2}{\sqrt{1-v^2/c^2}}\\
x' = \frac{x-vt/c^2}{\sqrt{1-v^2/c^2}}
\end{aligned}
\end{equation}

\end{enumerate}
\subsubsection{$\odot$ \textbf{Lorentz transformation}} % H3 title

\end{quote}
\begin{quote}
	Einstein: Two postitulates of S.R.
PI. All inertial observers are eqivalent.PII. The speed of light of light $c$ is the smae for all observers
\end{quote}

By PI, if a particle is free in $S$
\begin{equation}
\vec{r} = \vec{r}_0 + \vec{u} t \quad\Rightarrow\quad \vec{r}' = \vec{r}_0' + \vec{u}' t'
\end{equation}
so the transform is linear
\begin{equation}
\begin{pmatrix}
ct'\\ x'\\ y'\\ z'
\end{pmatrix}
= L \begin{pmatrix}
ct\\ x\\ y\\ z
\end{pmatrix},
\quad\text{ where } L =
\begin{pmatrix}
\gamma & -\gamma\beta & 0 & 0\\
-\gamma\beta & \gamma & 0 & 0\\
0 & 0 & 1 & 0\\
0 & 0 & 0 & 1\\
\end{pmatrix} \text{ and }
\gamma = \frac{1}{\sqrt{1-\beta^2}}, \quad \beta = \frac{v}{c}
\end{equation}
By PII, 

\underline{FIg9 - sphere}
\begin{equation}
\begin{aligned}
I(t,x,y,z) &= (ct)^2 - x^2 - y^2 - z^2 = 0\\
I(t',x',y',z') &= (ct')^2 - x'^2 - y'^2 - z'^2 = 0
\end{aligned}\quad\Rightarrow\quad I = I'
\end{equation}



From boost in $x$-direction, we have $(xt')^2 - x'^2 = (xt)^2 - x^2$

Introducing imaginary time $T = ict$, the equation become 
\begin{equation}
x^2+T^2 = x'^2 + T'^2,\quad \text{(Euclidean)}
\end{equation}
Now we can using a rotation to relate two coordinates
\begin{equation}
\begin{pmatrix}
x' \\ T'
\end{pmatrix}
=
\begin{pmatrix}
\cos\theta & \sin\theta \\
-\sin\theta & \cos\theta
\end{pmatrix}\begin{pmatrix}
x \\ T
\end{pmatrix}
\end{equation}
\underline{Fig10 - rotation of coordinates}
\begin{equation}
\vec{r} = x\hat{x} + T\hat{t} = x' \hat{x}' + T' \hat{t}'
\end{equation}
then check
\begin{equation}
\begin{aligned}
\hat{x}' = \cos\theta \hat{x} + \sin \theta \hat{t}\\
\hat{T}' = -\sin\theta \hat{x} + \cos \theta \hat{t}\\
\end{aligned}
\end{equation}
First we have $S'$ at rest $0'$ ($x'=0$) $\Rightarrow$ world line is $T'$-axis  $\Rightarrow$ $x\cos\theta + T\sin\theta=0$
\begin{equation}
v = \frac{x}{t} = \frac{icx}{T} = -ic\tan\theta
\end{equation}
Then we can get
\begin{equation}
\cos \theta = \frac{1}{\sqrt{1+\tan^2\theta}} = \frac{1}{\sqrt{1+(iv/c)^2}} = \frac{1}{\sqrt{1-v^2/{c^2}}} = \gamma
\end{equation}
and 
\begin{equation}
\sin \theta = \tan\theta \cdot \cos\theta = \gamma \cdot \frac{iv}{c}
\end{equation}
so the transformation become
\begin{equation}
\begin{pmatrix}
x' \\ T'
\end{pmatrix}
=
\begin{pmatrix}
\gamma & \gamma iv/c \\
-\gamma iv/c  & \gamma
\end{pmatrix}\begin{pmatrix}
x \\ T
\end{pmatrix}
\end{equation}
that is 
\begin{equation}
\begin{aligned}
x'
&= \cos\theta x + \sin \theta T = \cos\theta (x+\tan T)\\
&= \gamma \left(x + \frac{iv}{c}T\right) = \gamma (x-vt)\\
t' &= \gamma (t - vx/c^2)
\end{aligned}
\end{equation}


\begin{quote}
	Rmk:
\begin{enumerate}
	\item in $x-T$ system, L.T. can be viewedas a rotation with $\theta$ where $\tan\theta = iv/c$

	\item when $v\ll c$ , L.T. $\to$ G.T.

	\item L.T. $L( v )$ forms a group
	\item Identity $L(0) = \mathbb{I}$

	\item Inverse $L^{-1}(v) = L(-v)$ $\Rightarrow$ $L(v) L(-v) = \mathbb{I}$

	\item Closure $L(v')L(v) = L(v'')$: ex:

\begin{equation}
\begin{aligned}
\tan\theta'' &= \tan(\theta+\theta') = \frac{\tan\theta + \theta'}{1- \tan\theta \tan\theta'}\\
&= \frac{\frac{i(v+v')}{c}}{1+\frac{vv'}{c^2}} = \frac{iv''}{c}
\Rightarrow v'' = \frac{v" v'}{1+vv'/c^2}
\end{aligned}
\end{equation}

	\item associativity

\begin{equation}
L_3\cdot (L_2\cdot L_2) = (L_3\cdot L_2)\cdot L_1
\quad\Leftrightarrow\quad
\theta_3 + (\theta_2 + \theta_1) = \cdot (\theta_3 + \theta_2) + \theta_1
\end{equation}

\end{enumerate}
\subsubsection{$\odot$ Length contraction} % H3 title
measure the length in $S$ and $S'$ frameLet $\ell_0 = x'_B - x'_A$ at real in $S'$ called \textit{proper length}

\end{quote}
\underline{Fig11 - length contraction}

measure $x_A$ and $x_B$ in $S$ at the same time $t_A=t_B$, then
\begin{equation}
\ell = (x_B-x_A)
\end{equation}
and 
\begin{equation}
\begin{aligned}
x_A' = \gamma (x_A - vt)\\
x_B' = \gamma (x_B - vt)\\
\end{aligned}
\quad\Rightarrow\quad
(x_B'-x_A') = \gamma (x_B-x_A)
\quad\Rightarrow\quad
\ell_0 = \gamma \ell,\quad (\gamma>1\Rightarrow \ell<\ell_0)
\end{equation}


\subsubsection{$\odot$ time dialation} % H3 title

Let $t_0$ proper-time (co-miving time) in $S'$, which means the clock is moving with the obsever.

So that in $S$ we have the time interval
\begin{equation}
\Delta t = \gamma (\Delta t' + v\Delta '/c^2)
\quad\Rightarrow\quad
T = \gamma T_0 > T_0
\end{equation}


\begin{quote}
	Rmk experiment 1: the particle from the universe have more life time
\underline{Fig12 - particle from universe}
Rmk experiment 2: clock moving very fast
\end{quote}

\subsubsection{$\odot$ velocity} % H3 title

We define the velocity to be 
\begin{equation}
\vec{u} = \frac{d\vec{x}}{dt} = \left(u_x,u_y,u_z\right)
= \left(\frac{du_x}{dt},\frac{du_y}{dt},\frac{du_z}{dt}\right)
\end{equation}


The L.T. result that 
\begin{equation}
\begin{cases}
dt' = \gamma (dt - vdx/c^2)\\
dx' = \gamma (dx - vdt/c^2)\\
dy' = dy\\
dz' = dz
\end{cases}
\end{equation}
so the velocity 
\begin{equation}
u_x' = \frac{dx'}{dt'}
= \frac{\gamma (dx-vdt)}{\gamma (dt-vdx/c^2)}
= \frac{u_x-v}{1-vu_x/c^2}
\end{equation}
also 
\begin{equation}
u_y = \frac{u_y}{\gamma (1-u_xv/c^2)}
,\quad
u_z = \frac{u_z}{\gamma (1-u_xv/c^2)}
\end{equation}


\subsubsection{$\odot$ accerleration} % H3 title


\begin{equation}
{du_x} = \gamma^{-2} \frac{1}{(1+u_x'v/c^2)} du_x
,\quad
dt' = \gamma (dt-vdx/c^2)
\end{equation}

and 
\begin{equation}
dt = \gamma (dt' + vdx'/c^2) = \gamma (1+v\frac{dx'}{dt'}/c^2) dt'
= \gamma (1+vu_x/c^2) dt'
\end{equation}
solving that 
\begin{equation}
a_x = \frac{du_x}{dt} = \gamma^{-3} \frac{1}{\left(1+u_x'v/c^2\right)^3} a_x'
\end{equation}


\subsubsection{$\odot$ Simultaneity (同時性)} % H3 title

Simultaneity is not exist ? train 

\underline{Fig13 - train}

light cone

\underline{Fig14 - light cone}

\subsubsection{$\odot$ spacetime diagram  (real time)} % H3 title


\begin{equation}
\begin{aligned}
t' &= \gamma (t-vx/c^2) &&\to ct' = \gamma (ct-vx/c)\\
x' &= \gamma (x-vt) &&\to x' = \gamma \left(x-\frac{v}{c}(ct)\right)
\end{aligned}
\end{equation}

\begin{quote}
	Recall: in $S'$
\begin{itemize}
	\item $x'$-axis ($t'=0$) $\Rightarrow$ $\displaystyle ct=\left(\frac{v}{c}\right)x$
	\item $t'$-axis ($x'=0$) $\Rightarrow$ $\displaystyle x=\left(\frac{v}{c}\right)ct$

\end{itemize}
\underline{FIg15 - spacetime diagram}

\end{quote}
\begin{itemize}
	\item For $v>0$, folding the axis toward
	\item For $v<0$, expanding the axis outward

\end{itemize}
\underline{FIg16 - spacetime diagram simultaneity}

\begin{quote}
	Rmk:
\underline{Fig17 - contracton in space time diagram}
\end{quote}

\subsubsection{$\odot$ Calibration of length and time scale} % H3 title

\underline{Fig18 - hyperbola in space time diagram }

How to calibration rule and clock $\Rightarrow$ using invariant
\begin{equation}
x^2 - (ct)^2 = x'^2 - (ct')^2
\end{equation}




%================================================================================================
\end{document}
